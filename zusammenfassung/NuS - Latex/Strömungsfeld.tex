\documentclass[a4paper, 11pt]{scrartcl}


\makeatletter
\DeclareOldFontCommand{\rm}{\normalfont\rmfamily}{\mathrm}
\DeclareOldFontCommand{\sf}{\normalfont\sffamily}{\mathsf}
\DeclareOldFontCommand{\tt}{\normalfont\ttfamily}{\mathtt}
\DeclareOldFontCommand{\bf}{\normalfont\bfseries}{\mathbf}
\DeclareOldFontCommand{\it}{\normalfont\itshape}{\mathit}
\DeclareOldFontCommand{\sl}{\normalfont\slshape}{\@nomath\sl}
\DeclareOldFontCommand{\sc}{\normalfont\scshape}{\@nomath\sc}
\makeatother


\usepackage{mathptmx}
\usepackage{anyfontsize}
\usepackage{t1enc}
\usepackage[german]{babel} %choose your language
\usepackage[portrait, margin=1cm]{geometry}
\usepackage[utf8]{inputenc}
\usepackage[dvipsnames]{xcolor}
\usepackage{amscd, amsmath, amssymb, blindtext, empheq, enumitem, multicol, parskip}
\usepackage{graphicx}
\usepackage{tikz}
\usepackage{esint}
\usepackage{wrapfig}
\usepackage{setspace}
\usepackage{trfsigns}

% make document compact
\usepackage[compact]{titlesec}
\titlespacing{\section}{0pt}{*0}{*0}
\titlespacing{\subsection}{0pt}{*0}{*0}
\titlespacing{\subsubsection}{0pt}{*0}{*0}

\parindent 0pt
\pagestyle{empty}
\setlength{\unitlength}{1cm}
\setlist{leftmargin = *}
% Set the color of your style
% Avaiable are: Apricot, Aquamarine, Bittersweet, Black, Blue, blue, BlueGreen, BlueViolet, BrickRed, Brown, BurntOrange, CadetBlue, CarnationPink, Cerulean, CornflowerBlue, Cyan, Dandelion, DarkOrchid, Emerald, ForestGreen, Fuchsia, Goldenrod, Gray, Green, GreenYellow, JungleGreen, Lavender, ... (more at: http://en.wikibooks.org/wiki/LaTeX/Colors)
\def\StyleColor{Yellow}

\DeclareMathOperator{\rot}{rot}
\DeclareMathOperator{\divg}{div}


% define some colors
\usepackage{color}
\definecolor{section}{RGB}{221,95,50}
\definecolor{subsection}{RGB}{221,95,50}
\definecolor{subsubsection}{RGB}{225,157,41}
\definecolor{titletext}{RGB}{0,0,0}
\definecolor{formula}{RGB}{220,230,255}

% section color box
\setkomafont{section}{\mysection}
\newcommand{\mysection}[1]{%
    \Large\sf\bf%
    \setlength{\fboxsep}{0cm}%already boxed
    \colorbox{section}{%
        \begin{minipage}{\linewidth}%
            \vspace*{2pt}%Space before
            \leftskip2pt %Space left
            \rightskip\leftskip %Space right
            {\color{titletext} #1}
            \vspace*{1pt}%Space after
        \end{minipage}%
    }}
%subsection color box
\setkomafont{subsection}{\mysubsection}
\newcommand{\mysubsection}[1]{%
    \normalsize \sf\bf%
    \setlength{\fboxsep}{0cm}%already boxed
    \colorbox{subsection}{%
        \begin{minipage}{\linewidth}%
            \vspace*{2pt}%Space before
            \leftskip2pt %Space left
            \rightskip\leftskip %Space right
            {\color{titletext} #1}
            \vspace*{1pt}%Space after
        \end{minipage}%
    }}
    
%subsection color box
\setkomafont{subsubsection}{\mysubsubsection}
\newcommand{\mysubsubsection}[1]{%
    \normalsize \sf\bf%
    \setlength{\fboxsep}{0cm}%already boxed
    \colorbox{subsubsection}{%
        \begin{minipage}{\linewidth}%
            \vspace*{2pt}%Space before
            \leftskip2pt %Space left
            \rightskip\leftskip %Space right
            {\color{titletext} #1}
            \vspace*{1pt}%Space after
        \end{minipage}%
    }}    
%subsection color box
\title{Elektrisches, stationäres Strömungsfeld}
\author{www.n.ethz.ch/$\sim$zrene/nus1/nus1.html}
\date{}

\newcommand{\dis}[1]{\hspace{#1cm}}
    
% equation box        
\newcommand{\eqbox}[1]{\fcolorbox{section}{formula}{\hspace{0.5em}$\displaystyle#1$\hspace{0.5em}}}

%macro for vectors 
%\newcommand{\vect}[1]{\vec{#1}}
\newcommand{\vect}[1]{\boldsymbol{#1}}

%\renewcommand{\familydefault}{cmss}
\DeclareSymbolFont{letters}{OML}{ztmcm}{m}{it}
\DeclareSymbolFontAlphabet{\mathnormal}{letters}
%\everymath{\displaystyle}  %bigger equations
\begin{document}
%\setcounter{secnumdepth}{0} %no enumeration of sections
		\maketitle
		\setcounter{section}{3}
				\subsection{Strom}
				
				$\displaystyle{I=\frac{dQ}{dt}=\iint_A{\vec{J}\cdot}d\vec{A},}$ \dis{0.2} $[I]=A,$ \dis{0.2} $\displaystyle{J=\frac{dI}{dA},}$ \dis{0.2} $\displaystyle{[J]=\frac{A}{m^2}}$\\
				Stat. Strömungsfeld, wenn $I$ konst.: $\varoiint_{A}{\vec{J}\cdot}d\vec{A} = 0$ \\
				\begin{itemize}
					\item \textbf{Spezifische Leitfähigkeit:}\\
					Driftgeschw. $\vec{v}_{Drift}=-\mu_e\vec{E}$ wobei $\mu_e=$ "Beweglichkeit"\\
					$\vec{J}=\vec{V}_{Drift}\rho=\vec{v}nq=\underbrace{-\rho\mu_e}_\kappa\vec{E},$ \dis{0.2} $\kappa=$spez.Leitf.,
					$[\kappa]=\frac{A}{Vm}=\frac{1}{\Omega m}$
					\item \textbf{Spezifischer Widerstand:} $\rho_R=\frac{1}{\kappa},$ $[\rho_R]=\Omega m=\frac{Vm}{A}$
					\item \textbf{Temperaturabhängigkeit:}\\
					$\rho_R(T)=\rho_{R,20^\circ C}\left(1+\alpha(T-20^\circ C)\right)$
					\item \textbf{Ohmsches Gesetzt:} \eqbox{U=R\cdot I}, $[R]=\frac{V}{A}=\Omega$\\
					$\vec{J}=\kappa\vec{E},$\dis{0.2} $R=\frac{U}{I}=\frac{l}{\kappa A}=\frac{\rho_R l}{A}=\frac{\int_S{\vec{E}\cdot}d\vec{s}}{\kappa\iint_A{\vec{E}\cdot}d\vec{A}}$
					\item \textbf{Leitwert:} \eqbox{G=\frac{1}{R}} $[G]=S$ (Siemens)
				\end{itemize}
				
				\subsection{Sprungstellen bei Materialübergängen }
				\begin{itemize}
					\item \textit{Normalkomponenten.} \eqbox{J_{n1}=J_{n2}},\eqbox{\kappa_1E_{n1}=\kappa_2E_{n2}}\\
					Die Normalkomponente der Stromdichte ist stetig.
					\item \textit{Tangentialkomp.:} \eqbox{E_{t1}=E_{t2}},\eqbox{\frac{J_{t1}}{J_{t2}}=\frac{\kappa_1}{\kappa_2}}\\
					Die Tangentialkomponente des E-Feldes ist stetig.
				\end{itemize}
				\subsection{Energie und Leistung (1-102)}
				$\displaystyle{W_e=\int_{0}^{t}{P(\tau)}d\tau}$ und $P(t)=\frac{dW_e}{dt}$\\$P=UI=I^2R=U^2/R$\\
				\textbf{Verlustleistungsdichte:} $p_V=\frac{dP}{dV}=\vec{E}\cdot\vec{J}$\\
				$P=\iiint_V{p_V}dV=\iiint_V{\vec{E}\cdot\vec{J}}dV$
			
\setcounter{secnumdepth}{2}
\end{document}
