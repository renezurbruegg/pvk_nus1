%----------------------------------------------------------------
%
%  File    :  thesis-style.tex
%
%  Author  :  Keith Andrews, IICM, TU Graz, Austria
%
%  Created :  27 May 93
%
%  Changed :  19 Feb 2004
%
% styling and technical implementation adopted 2011 by Karl Voit
%----------------------------------------------------------------
\newpage
\section{Zeitlich veränderliches Magnetfeld}
\subsection{Induktion}

%TODO
einführung

\gl{Gleichung}{Induktionsgesetz}
\begingl
\formulaBegin
  $\displaystyle \oint_s \vec{E}\cdot d\vec{s} = -\frac{d}{dt} \big ( \iint_{A_s} \vec{B} \cdot d\vec{A} \big )$
\formulaEnd
\textbf{Variablen} \\
$B = $ Magnetische Flussdichte $[B] = T$\\
$A_s = $ Vom Weg S aufgespannte Fläche $[A_s] = m^2$ \\

Falls B Feld Konstant auf Fläche und senkrecht:

\formulaBegin
  $\displaystyle  \oint_s \vec{E}\cdot d\vec{s} = -\frac{d}{dt} \big ( B_{eff} \cdot A_s \big ) $
\formulaEnd
\iend

\definition{Bewegungsinduktion}
\beginip
%TODO
\iend

\bsptask{Beispiel}{Bewegungsinduktion}
\beginbsp
%todo
TODO
\iend

\definition{Selbstinduktion}
\beginip
%TODO
todo
\iend


\definition{Gegentinduktion}
\beginip
%TODO
todo
\iend


\subsection{Charakteristische Gleichungen von Kapazität und Induktivität}
\gl{Gleichung}{Induktivität}
\begingl
%TODO
todo
\iend

\gl{Gleichung}{Kondensator}
\begingl
Der Zusammenhang zwischen Strom und Spannung am Kondensator ist wie folgt gegeben
\formulaBegin
$\displaystyle u_c(t) = \frac{1}{C} \int_0^t i_c(t) dt$ \\

$\displaystyle i_c(t) = c \cdot \frac{d}{d t} (u_c)$ \\
\formulaEnd
\iend


\textbf{Begründung} \\
Mit dem Wissen, dass Strom definiert ist als die Ladung pro Zeit $ \displaystyle \frac{dQ}{dt} = I$ folgt folgendes: \\
\begin{center}
	$\displaystyle \frac{\partial}{\partial t} (C \cdot U) = \frac{\partial}{\partial t} (Q) $ \\
	$ \displaystyle C  \cdot \frac{dU}{dt} = I \rightarrow U(t) = \frac{1}{C} \cdot \int_0^t I \cdot dt$
\end{center}
