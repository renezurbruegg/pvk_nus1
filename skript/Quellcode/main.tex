\newcommand{\mypapersize}{A4}
%% e.g., "A4", "letter", "legal", "executive", ...
%% The size of the paper of the resulting PDF file.

\newcommand{\mylaterality}{oneside}
%% "oneside" or "twoside"
%% Either you are creating a document which is printed on both, left pages
%% and right pages (twoside) or you create a document which is printed
%% on right pages only (oneside).

\newcommand{\mydraft}{false}
%% "true" or "false"
%% Use draft mode? If true, included graphics are replaced by empty
%% rectangles (of same size) and overfull boxes (in margin space) are
%% marked with black box (-> easy to spot!)

\newcommand{\myparskip}{half}
%% e.g., "no", "full", "half", ...
%% How to separate paragraphs: indention ("no") or spacing ("half",
%% "full", ...).

\newcommand{\myBCOR}{0mm}
%% Inner binding correction. This value depends on the method which is
%% being used to bind your printed result. Some techniques do not
%% require a binding correction at all ("0mm"), other require for
%% example "5mm". Refer to KOMA script documentation for a detailed
%% explanation what a binding correction is and how to measure it.

\newcommand{\myfontsize}{12pt}
%% e.g., 10pt, 11pt, 12pt
%% The font size of the main text in pt (points).

\newcommand{\mylinespread}{1.0}
%% e.g., 1.0, 1.5, 2.0
%% Line spacing in %/100. For example 1.5 means 150% of the usual line
%% spacing. Please use with caution: 100% ("1.0") is fine because the
%% font was designed for it.

\newcommand{\mylanguage}{english, ngerman}
%% "english,ngerman", "ngerman,english", ...
%% NOTE: The *last* language is the active one!
%% See babel documentation for further details.

%% BibLaTeX-settings: (see biblatex reference for further description)
\newcommand{\mybiblatexstyle}{authoryear}
%% e.g., "alphabetic", "authoryear", ...
%% The biblatex style which is being used for referencing. See
%% biblatex documentation for further details and more values.
%%\usepackage{mathtools}
%% CAUTION: if you change the style, please check for (in)compatible
%%          "biblatex" package options in the file
%%          "template/preamble.tex"! For example: "alphabetic" does
%%          not have an option "dashed=..." and causes an error if it
%%          does not get removed from the list of options.

\newcommand{\mybiblatexdashed}{false}  %% "true" or "false"
%% If true: replace recurring reference authors with a dash.

\newcommand{\mybiblatexbackref}{true}  %% "true" or "false"
%% If true: create backward links from reference to citations.

\newcommand{\mybiblatexfile}{references-biblatex.bib}
%% Name of the biblatex file that holds the references.

\newcommand{\mydispositioncolor}{30,103,182}
%% e.g., "30,103,182" (blue/turquois), "0,0,0" (black), ...
%% Color of the headings and so forth in RGB (red,green,blue) values.
%% NOTE: if you are using "0,0,0" for black, printers might still
%%       recognize pages as color pages. In case this is a problem
%%       (paying for color print-outs vs. paying for b/w-printouts)
%%       please edit file "template/preamble.tex" and change
%%       "\definecolor{DispositionColor}{RGB}{\mydispositioncolor}"
%%       to "\definecolor{DispositionColor}{gray}{0}" and thus
%%       overwriting the value of \mydispositioncolor above.

\newcommand{\mycolorlinks}{true}  %% "true" or "false"
%% Enables or disables colored links (hyperref package).

\newcommand{\mytitlepage}{template/title_Diplomarbeit_KF_Uni_Graz}
%% Your own or one of following pre-defined title pages:
%% "template/title_plain_maketitle": simple maketitle page
%% "template/title_Diplomarbeit_KF_Uni_Graz.tex": fancy (german) title page for KF Uni Graz
%% "template/title_Thesis_TU_Graz":
%%             titlepage for Graz University of Technology (correct
%%             (old?) Corporate Design) by Karl Voit (2012)
%% "template/title_Thesis_TU_Graz_-_kazemakase":
%%             titlepage for Graz University of Technology
%%             (correct new Corporate Design) by kazemakase (2013):
%%             see https://github.com/novoid/LaTeX-KOMA-template/issues/5
%% "template/title_VWA": titlepage for Vorwissenschaftliche Arbeit

\newcommand{\mytodonotesoptions}{}
%% e.g., "" (empty), "disable", ...
%% Options for the todonotes-package. If "disable", all todonotes will
%% be hidden (including listoftodos).

%% Load main settings for document preamble:
\input{template/preamble}%% DO NOT REMOVE THIS LINE!

\setboolean{myaddcolophon}{true}  %% "true" or "false"
%% If set to "true": a colophon (with notes about this document
%% template, LaTeX, ...) is added after the title page.
%% Please do not set to "false" without a good reason. The colophon
%% helps your readers to get in touch with LaTeX and to find this template.

\setboolean{myaddlistoftodos}{false}  %% "true" or "false"
%% If set to "true": the current list of open todos is added after the
%% table of contents. If \mytodonotesoptions is set to "disable", no
%% list of todos is added, independent of this setting here.

\setboolean{english_affidavit}{true}  %% "true" or "false"
%% If set to "true": the language of the statutory declaration text is set to
%% English, otherwise it is in German.

\definecolor{bspcolor}{rgb}{0.202,0.5,0.22}
\definecolor{glcolor}{rgb}{0.717,0.1,0.1}
%% ========================================================================
%%%% Document metadata
%% ========================================================================

%% general metadata:
\newcommand{\myauthor}{René Zurbrügg}  %% also used for PDF metadata (hyperref)
\newcommand{\myauthorwithexistingtitles}{\myauthor{}, OLDDEGREE}  %% including
%% university degree already held
%% (BSc, MSc, ...)
\newcommand{\mytitle}{Netzwerke und Schaltungen 1}  %% also used for PDF metadata (hyperref)
\newcommand{\mysubtitle}{ }  %% only used with title_Thesis_TU_Graz_-_kazemakase
\newcommand{\mysubject}{SUBJECT}  %% also used for PDF metadata (hyperref)
\newcommand{\mykeywords}{KEYWORDS}  %% also used for PDF metadata (hyperref)

%% this information is used only for generating the title page:
\newcommand{\myworktitle}{Master's Thesis}  %% official type of work like ``Master theses''
\newcommand{\mygrade}{Master of Science} %% title you are getting with this work like ``Master of ...''
\newcommand{\mystudy}{Telematik} %% your study like ``Arts''
\newcommand{\mydegreeprogramme}{Master's degree programme: \mystudy} %% Master's or PhD degree programme
\newcommand{\myuniversity}{Graz University of Technology} %% your university/school
\newcommand{\myfaculty}{ }  %% only used with title_Thesis_TU_Graz_-_kazemakase
\newcommand{\myinstitute}{Institute for Softwaretechnology} %% affiliation
\newcommand{\myinstitutehead}{Univ.-Prof.\,Dipl-Ing.\,Dr.techn.~Some One} %% head of institute
\newcommand{\mysupervisor}{Dr.~Some Body} %% your supervisor
\newcommand{\mycosupervisor}{\ }  %% only used with title_Thesis_TU_Graz_-_kazemakase
\newcommand{\myevaluator}{Prof.~Some Genius} %% your evaluator
\newcommand{\myhomestreet}{Street~42} %% your home street (with house number)
\newcommand{\myhometown}{Zürich} %% your home town
\newcommand{\myhomepostalnumber}{8010} %% your postal number of home town
\newcommand{\mysubmissionmonth}{Dezember} %% month you are handing in
\newcommand{\mysubmissionyear}{2018} %% year you are handing in
\newcommand{\mysubmissiontown}{\myhometown} %% town of handing in (or \myhometown)


\newcounter{bspcounter}



\def\doubleunderline#1{\underline{\underline{#1}}}


%% defined an anvironment for the style Keith used to use:
\newenvironment{mykeithtabbing}[1]{%%
	\begin{tabular}{lp{0.9\hsize}}
		}{%%
	\end{tabular}
}


\newcommand{\formulaBegin}{

	\begin{tcolorbox}
	\begin{center}
}

\newcommand{\formulaEnd}{
	\end{center}
	\end{tcolorbox}
}

\newcommand{\fspace}{
	\vspace{2mm}
}

\newcommand{\fix}{

	\vspace{-2mm}
}


	\newcommand{\beginbsp}{
		\hfill {\color[rgb]{0.202,0.5,0.22}{\vrule width 2pt }}\hfill
		\begin{minipage}{0.95\textwidth}

			\begin{addmargin}[0.5em]{0cm}
				}


	\newcommand{\beginip}{
		\hfill {\color[rgb]{0.117,0.403,0.713}{\vrule width 2pt }}\hfill
		\begin{minipage}{0.95\textwidth}
			\begin{addmargin}[0.5em]{0cm}
				}



	\newcommand{\beginvor}{
		\beginip
	}
	\newcommand{\begingl}{

		\hfill {\color[rgb]{0.717,0.1,0.1}{\vrule width 2pt }}\hfill
		\begin{minipage}{0.95\textwidth}
			\begin{addmargin}[0.5em]{0cm}
				}

	\newcommand{\iend}{
		\end{addmargin}
		\end{minipage}
	}

	\newcommand{\definition}[1]{
		%\PencilRightDown \vspace{-1,0cm}
		\important{Definition}{#1}

	}



	\newcommand{\gleichung}[2]{
		%\PencilRightDown \vspace{-1,0cm}
		\color{red}
		\subparagraph{#1}  \textbf{#2}

	}




	\newcommand{\vorgehen}[2]{
		%\PencilRightDown \vspace{-1,0cm}
		\subparagraph{  \hspace{-0.85cm} \textcolor{black}{\Checkmark \ } #1}  \textbf{#2}

	}

	\newcommand{\bsptask}[2]{
		%\PencilRightDown \vspace{-1,0cm}
		\subparagraph{  \hspace{-0.85cm} \textcolor{black}{\PencilRightDown \ } \textcolor{bspcolor}{#1 \thebspcounter}}
		\stepcounter{bspcounter}  \textbf{#2}

	}
	\newcommand{\bsp}[2]{
		%\PencilRightDown \vspace{-1,0cm}
		\subparagraph{\textcolor{bspcolor}{#1}}  \textbf{#2}

	}


	\newcommand{\gl}[2]{
		%\PencilRightDown \vspace{-1,0cm}
		\subparagraph{\textcolor{glcolor}{#1}}  \textbf{#2}

	}
	\newcommand{\important}[2]{
		%\PencilRightDown \vspace{-1,0cm}
		\subparagraph{#1}  \textbf{#2}

	}
	\newcommand{\mybadgood}[2]{%%
		\begin{mykeithtabbing}
			{}\emph{Bad:}  & \sout{#1}  \\
			\emph{Good:}   & #2  \\
		\end{mykeithtabbing}
	}

	\newcommand{\ibox}[1] {
		\tcbox[sharp corners, boxsep=1mm, boxrule=0.2mm,
		colframe=black!30!black, colback=white]{#1}
	}


	%% additional information for generic_documentation title page
	\newcommand{\myid}{1234567} %% Matrikelnummer
	\newcommand{\mylecture}{LECTURE} %%


	%% ========================================================================
	%%%% MISC command def		initions
	%% ========================================================================
	\input{template/mycommands}

	%% ======================	==================================================
	%%%% Typographic settings
	%% ========================================================================
	\input{template/typographic_settings}


	%% ========================================================================
	%%%% MISC usepackages
	%% ========================================================================

	\usepackage{wasysym}

	\usepackage{mathtools}
	\usepackage{dingbat,bbding}
	\usepackage[bottom=1in,top=0.5in,left=2cm,right=2cm]{geometry}
	\usepackage{tcolorbox}

	\usepackage{xhfill}
	% \usepackage{mathptmx}
	\usepackage{pslatex} %Times font
	\usepackage[font={small,sf},format=plain,labelfont=bf,up]{caption}
	%\usepackage[top=1.5cm, bottom=1.5cm, outer=5cm, inner=2cm, heightrounded, marginparwidth=2.5cm, marginparsep=2cm]{geometry}
	%%%% MISC self-defined commands and settings
	%% ========================================================================

	%% ... it's OK to put here your own newcommand/newenvironment-definitions ...
	\usepackage{float}
	\usepackage{subfigure}


	\usepackage{multicol}
	\newcommand{\myLaT}{\LaTeX{}@TUG\xspace} %% LaTeX@TUG text "logo"

	\newcommand{\myimportant}{%% mark important chaptersfbsp
		\marginpar{\vspace{-1em}\rightpointleft}
	}
	\newcommand{\myinteresting}{\marginpar{\vspace{-2em}\PencilLeftDown}}

	\hyphenation{ex-am-ple hy-phen-ate}  %% in order to use German umlauts
	%% here (Ver-\"of-fent-li-chung), you have to check for
	%% activated \usepackage[T1]{fontenc} in the preamble

	%% override default language of babel: (be sure to know, what you're
	%% doing here)
	%\selectlanguage{american}
	%\selectlanguage{ngerman}

	%% ========================================================================
	%%%% Templates
	%% ========================================================================

	%% template for inserting figures:
	% \myfig{}%% filename
	%       {}%% width/height
	%       {}%% caption
	%       {}%% optional (short) caption for list of figures
	%       {fig:}%% label

	%% acronyms in small caps: \myacro{UNESCO}


	\input{template/pdf_settings}  %% should be *last* definitions in preamble!
	%% ========================================================================
	%%%% begin{document}
	%% ========================================================================
	\begin{document}

	%\frontmatter                    %% KOMA: roman page numbers and such; only available in scrbook

	\input{colophon}                %% defines information about editor, LaTeX, font, ...

	%% Choose your desired title page:
	\input{\mytitlepage}            %% include title page



	\begin{center}
		\LARGE\textbf{Vorwort}
	\end{center}
	Dieses Skript entstand als Beiblatt zum Prüfungsvorbereitungskurs \texttt{"}Netzwerk und Schaltungen 1\texttt{"} im Januar 2019.
	Es ist primär als Hilfe für die Studenten gedacht, mit dem Ziel, den Fokus auf die prüfungsrelevanten Themen zu legen.
	Fragen, Anregungen und vor allem Korrekturen sind willkommen und sollten an zrene@student.ethz.ch adressiert werden. \\
	Die neuste Version des Skriptes befindet sich immer auf: \texttt{n.ethz.ch/~zrene/nus1-pvk}
	\\
	Dieses Skript wurde von Studenten geschrieben und kann Fehler enthalten.
	\begin{center}
		Version 1.2.0
	\end{center}
	\begin{center}
		\hspace{1cm}
		\includegraphics[scale=0.48]{img/version.png}
	\end{center}


	%% include the abstract without chapter number but include it on table of contents:
	\cleardoublepage
	\phantomsection

	\tableofcontents                %% this produces the table of contents - you might have guessed :-)

	\newpage
	%\mainmatter                     %% KOMA: marks main part using arabic page numbers and such; only available in scrbook


	 %----------------------------------------------------------------
%
%  File    :  thesis-style.tex
%
%  Author  :  Keith Andrews, IICM, TU Graz, Austria
%
%  Created :  27 May 93
%
%  Changed :  19 Feb 2004
%
% styling and technical implementation adopted 2011 by Karl Voit
%----------------------------------------------------------------

					\section{Mathematische Grundlagen}
					\label{chap:Style}

					\subsection{Koordinatensysteme}

					Häufig sind wir daran interessiert, geometrische Dinge in der Mathematik darzustellen um mit ihnen zu rechnen.
					\important{Karthsische Koordinaten} {}
					\beginip
					\begin{center}

					\includegraphics[scale = 0.5	]{kugelkoord.png}

				\end{center}
					\iend

					\formulaBegin
						$\vec{F}(r, \theta, \varphi) = \left(\begin{array}{c} 1 \\ 0 \\ 0 \end{array}\right)_{Kugel\ Koordinaten} = 1 \cdot \vec{e}_r$ \\

					\formulaEnd
					\begin{center}

					\includegraphics[scale=0.4]{spherical_r.png}

					\end{center}


					\formulaBegin
						$\vec{F}(r, \theta, \varphi) = \left(\begin{array}{c} 0 \\ 1 \\ 0 \end{array}\right)_{Kugel\ Koordinaten} = 1 \cdot \vec{e}_\theta$ \\

					\formulaEnd
					\begin{center}

					\includegraphics[scale=0.4]{spherical_theta.png}

					\end{center}


					\formulaBegin
						$\vec{F}(r, \theta, \varphi) = \left(\begin{array}{c} 0 \\ 0 \\ 1 \end{array}\right)_{Kugel\ Koordinaten} = 1 \cdot \vec{e}_\varphi$ \\

					\formulaEnd
					\begin{center}

					\includegraphics[scale=0.4]{spherical_phi.png}

					\end{center}









					\formulaBegin
						$\vec{F}(\rho, \varphi, z) = \left(\begin{array}{c} 1 \\ 0 \\ 0 \end{array}\right)_{Zylinder\ Koordinaten} = 1 \cdot \vec{e}_\rho$ \\

					\formulaEnd
					\begin{center}

					\includegraphics[scale=0.4]{zylindric_rho.png}

					\end{center}


					\formulaBegin
						$\vec{F}(\rho, \varphi, z) = \left(\begin{array}{c} 0 \\ 1 \\ 0 \end{array}\right)_{Zylinder\ Koordinaten} = 1 \cdot \vec{e}_\varphi$ \\

					\formulaEnd
					\begin{center}

					\includegraphics[scale=0.4]{zylindric_phi.png}

					\end{center}




					\formulaBegin
					$\vec{F}(\rho, \varphi, z) = \left(\begin{array}{c} 0 \\ 0 \\ 1 \end{array}\right)_{Zylinder\ Koordinaten} = 1 \cdot \vec{e}_z$ \\

					\formulaEnd
					\begin{center}

					\includegraphics[scale=0.4]{zylindric_z.png}

					\end{center}



					\important{Zylinderkoordinaten} {}
					\beginip

					\includegraphics[scale = 0.4]{zylinderkoord.png}
					\iend

					\important{Kugelkoordinaten} {}
					\beginip
					\iend




					\important{Defintion}{Vektorfeld}
					\beginip
						Ein Vektorfeld, bezeichnet eine mathematische Funktion, welche anstatt einer skalaren Grösse, Vektoren zurück gibt. \\
						Häufig ist ein Vektorfeld nicht nur von einer Variable (x) abhängig, sondern besitzt 3 Parameter (x,y,z) die wir manchmal als Ortsvektor $\vec{r}$ zusammenfassen.
					\iend


					\important{Defintion}{Wegintegral über ein Vektorfeld}
					\beginip
					% TODO %

					\iend


					\important{Defintion}{Fluss eines Vektorfeldes}
					\beginip
							\formulaBegin
							$ \oiint_A \vec{E} \cdot d\vec{A}$
							\formulaEnd
							Bezeichnen wir als Oberflächenintegral des Vektorfeldes $\vec{E}$.  Der Wert des Integrales gibt uns eine Information darüber, wieviel Feld durch eine gegebene Hüllfäche "fliesst"

					\iend

					\important{Beispiel}{}
					\beginip
					Wir betrachten ein Rohr der Länge L und mit radius R. \\
					In der Mitte des Rohres im Uhrsprung des Koordinatensystems befindet sich eine Wasserquelle, welche eine gewisse Menge Wasser generiert. \\
					Die Flussdichte des Wassers, ist durch Folgendes Vektorfeld gegeben: \\
					\begin{center}

					$  \vec{J}_{wasser}(x,y,z) = \left\{
					       \begin{array}{@{}l@{\thinspace}l}
					          1 \cdot \vec{e}_y  \cdot [\frac{L}{s \cdot m^2}] &: \frac{-l}{2} < y < \frac{l}{2}\ und\ (x^2 + z^2) \leq R \\
					          0 \cdot \vec{e}_y  \cdot [\frac{L}{s \cdot m^2}] &: sonst\\
					       \end{array}
					       \right$
	 					\end{center}

						\begin{enumerate}
							\item  Zeichne das Vektorfeld in einem 3-Dimensionalen Koordinatensystem. \\
							\includegraphics[scale=0.4]{plot.png}
							\includegraphics[scale=0.4]{plot2.png}
							\item
						\end{enumerate}
					\iend

	 \newpage
	 \section{Elektrostatik}
\label{chap:Style}


Elektrisch geladene Teilchen, welche sich in der Nähe eines anderen, geladenen Teilchen befinden, verspüren eine Kraftwirkung, die abhängig der eigenen Ladung ist. \\
Um diese Kraftwirkung zu beschreiben, wurde der Begriff des \textbf{elektrischen Feldes} eingeführt.

\definition{Elektrisches Feld}
\beginip
Das elektrische Feld, beschreibt die Kraftwirkung auf geladene Teilchen im Raum. \\
Es ordnet jedem Punkt im Raum einen Vektor $\vec{E}$ zu, der in die Richtung der Kraftwirkung zeigt.
Für die Kraftwirkung auf ein Teilchen mit der Ladung Q und dem Ortsvektor $\vec{r}$ gilt: \\
\formulaBegin
$\vec{F} = Q \cdot \vec{E}(\vec{r})$
\formulaEnd
\iend
\fspace
\textbf{Einige wichtige Felder:}
\bsp{Beispiel}{E-Feld einer Punktladung}
\beginbsp
Das elektrische Feld einer Punktladung Q ist gegeben als:
\formulaBegin
$\displaystyle \vec{E} = \frac{1}{4 \pi \epsilon} \cdot \frac{Q}{r^2}\cdot \vec{e}_r$
\formulaEnd
\begin{center}
	\ibox{\includegraphics[scale=0.45]{img/e-feld-pkt.png}}
\end{center}
\iend

\newpage


\bsp{Beispiel}{E-Feld einer Platte}
\beginbsp
Das elektrische Feld einer Platte mit Ladung Q ist gegeben als:
\formulaBegin
%Todo add Koordinatensystem
$\displaystyle \vec{E} = \begin{cases}
\frac{Q}{2\cdot A \varepsilon} \cdot \vec{e}_n & , Rechts \\
\frac{-Q}{2\cdot A \varepsilon} \cdot \vec{e}_n & , Links\\

\end{cases}
$
\formulaEnd
\begin{center}
	\ibox{\includegraphics[scale=1.4]{img/e-feld-platte.png}}
\end{center}
\iend



\bsp{Beispiel}{E-Feld eines Kondensators}
\beginbsp
Das elektrische Feld in einem Kondensator mit der Ladung Q oder Spannung U ist gegeben als:
\formulaBegin
$\displaystyle \vec{E} = \frac{Q}{A\varepsilon} \cdot \vec{e}_d = \frac{U}{d	} \vec{e}_d$
\formulaEnd
\begin{center}
	\ibox{\includegraphics[scale=0.45]{img/e-feld-kond.png}}
\end{center}
\iend
\newpage


\subsection{(Ladungs-)Dichte}
In der Physik sind wir häufig nicht nur daran interessiert, wieviele Ladungsträger sich in einem Volumen befinden, sondern wir möchten gerne Informationen darüber haben, wie sie geometrisch angeordnet sind. (= Verteilen sich alle Ladungsträger auf der Oberfläche, sind alle im Mittelpunkt zentriert etc). \\
Dazu werden \textbf{Dichtefunktionen} verwendet.

\definition{Linienladungsdichte $\lambda(x)$}
\beginip
Eine \textbf{Linienladungsdichte} gibt an, wie sich Ladungen entlang einer Linie anordnen. \\
Sind die Ladungsträger \textbf{gleichmässig} auf einer Linie verteilt, so ist die Dichte einfach eine konstante mit Einheit $\frac{C}{m}$. \\
Möchten wir die Gesamtladung auf Basis einer Dichte berechnen, so müssen wir die Dichte integrieren: \\
\formulaBegin
$\displaystyle Q_{AB} = \int_A^B \lambda (x) \cdot dx$
\formulaEnd
\iend

\bsptask{Beispiel}{Linienladungsdichte}
\beginbsp
Eine Gesamtladung von $Q=10C$ verteilt sich gleichmässig auf einer Gerade der Länge $l = 2m$. \\
\begin{itemize}
	\item 1) Geben Sie die Linienladungsdichte $\lambda(x)$ im Bereich $ 0 < x <l$ an.
	\item 2) Geben Sie die Gesamtladung im eingeschlossenem Bereich $ 0 < x < l/3$ an.
\end{itemize}
\iend

\newpage
\bsp{Lösung}{}
\beginbsp
\begin{itemize}
	\item 1) Da sich die Ladungen gleichmässig verteilen, ist die Ladungsdichte unabhängig vom Ort x
	      \begin{center}
	      	$\lambda(x) = \lambda = \frac{C}{l} = 5 \frac{C}{m}$
	      \end{center}
	\item 2) Die eingeschlossene Ladung berechnet sich grundsätzlich mittels Integration:
	      \begin{center}
	      	$\displaystyle Q = \int_0^{l/3} \lambda(x) \cdot dx = \int_0^{l/3} 5 \frac{C}{m} \cdot dx = 5 \frac{C}{m} \cdot \frac{l}{3} = \frac{10}{3} C$
	      \end{center}
\end{itemize}
\iend









\definition{Flächenladungsdichte $\sigma(x,y)$}
\beginip
Eine \textbf{Flächenladungsdichte} gibt an, wie sich Ladungen auf einer Fläche anordnen. \\
Sind die Ladungsträger \textbf{gleichmässig} auf der Fläche verteilt, so ist die Dichte einfach eine konstante mit Einheit $\frac{C}{m^2}$. \\
Möchten wir die Gesamtladung auf Basis einer Dichte berechnen, so müssen wir die Dichte integrieren: \\
\formulaBegin
$\displaystyle Q_{A} = \iint_A \sigma (x,y) \cdot dA$
\formulaEnd
Ist die Dichte konstant folgt:
\formulaBegin
$\displaystyle Q_{A} = \sigma \cdot A$
\formulaEnd
\iend







\bsptask{Beispiel}{Flächenladungsdichte}
\beginbsp
Eine Gesamtladung von $Q$ verteilt sich gleichmässig auf der Oberfläche einer Kugel mit Radius R. \\
Wie gross ist die Flächenladungsdiche $\sigma$ auf der Kugeloberfläche?
\iend


\bsp{Lösung}{}
\beginbsp
Da sich die Ladungen gleichmässig verteilen, gilt für die Ladungsdichte:
\begin{center}
	$\displaystyle \sigma = \frac{Q}{A} = \frac{Q}{4\pi R^2}$
\end{center}
\iend






\newpage

\definition{Volumenladungsdichte $\rho(x,y,z)$}
\beginip
Eine \textbf{Volumenladungsdichte} gibt an, wie sich Ladungen in einem Volumen anordnen. \\
Sind die Ladungsträger \textbf{gleichmässig} im Volumen verteilt, so ist die Dichte einfach eine konstante mit Einheit $\frac{C}{m^3}$. \\
Möchten wir die Gesamtladung auf Basis einer Dichte berechnen, so müssen wir die Dichte integrieren: \\
\formulaBegin
$\displaystyle Q_{V} = \iiint_V \rho (x,y,z) \cdot dV$
\formulaEnd
Ist die Dichte konstant folgt:
\formulaBegin
$\displaystyle Q_{V} = \rho \cdot V$
\formulaEnd
\iend




\bsptask{Beispiel}{Volumenladungsdichte}
\beginbsp
Eine Hohlkugel mit Innenradius $R_1$ und Aussenradius $R_2$ sei im Bereich $ R_1 \leq r \leq R_2$ gleichmässig mit der Ladung $Q$ gefüllt. \\
Gesucht ist die Volumenladungsdichte $\rho(r,\theta,\varphi)$ in Kugelkoordinaten, welche die Ladungsdichte im gesamten Raum beschreibt.
\iend


\bsp{Lösung}{}
\beginbsp
Da sich die Ladungen nur in einem gewissen Teil des Raumes befinden, müssen wir eine Fallunterscheidung durchfüren:
\begin{enumerate}
	\item $\mathbf{r < R_1}$ \\
	      Hier befindet sich keine Ladung, somit ist auch die Volumenladungsdichte gleich 0
	\item $\mathbf{R_1 \leq r \leq R_2}$ \\
	      Hier ist die Ladung gleichmässig auf dem Volumen $V = 4\cdot \pi (R_2^3 - R_1^3)$ verteilt, somit beträgt die Volumenladungsdichte:
	      \begin{center}
	      	$\displaystyle \rho = \frac{Q}{V} = \frac{Q}{4 \cdot \pi (R_2^3 - R_1^3) }$
	      \end{center}
	\item $\mathbf{r > R_3}$ \\
	      Hier befindet sich keine Ladung, somit ist auch die Volumenladungsdichte gleich 0
\end{enumerate}

Somit gilt für die ortsabhängige Volumenladungsdichte:
\begin{center}
	$\displaystyle	\rho(r,\theta,\varphi) =
	\begin{cases}
		0                                     & r < R_1             \\
		\frac{Q}{4 \cdot \pi (R_2^3 - R_1^3)} & R_1 \leq r \leq R_2 \\
		0                                     & r > R_2             \\
	\end{cases}$
\end{center}

\iend


\newpage

\subsection{Elektrische Flussdichte}
Quelle des elektrischen Feldes sind geladene Teilchen. Jedoch hängt das elektrische Feld davon ab, in was für einem Material wir uns befinden. \\
Das elektrische Feld in einem Metall ist zum Beispiel kleiner, als das elektrische Feld im Vakuum. \\
Grund dafür ist die \textbf{elektrische Influenz}, welche das ursprüngliche Feld abschwächt. \\
Um gewisse Rechnungen zu vereinfachen, definieren wir ein neues Feld.

\important{Definition}{Die elektrische Flussdichte}
\beginip
Die elektrische Flussdichte $\vec{D}$ beschreibt das Feld, welches existieren würde, falls kein Material vorhanden wäre. \\
Quelle der elektrischen Flussdichte ist die elektrische Ladung Q.
\iend

Da die elektrische Flussdichte  $\vec{D}$ ausschliesslich von den Ladungsträger abhängt,
gibt es eine Formel, die die beiden Grössen in Verbindung bringt:

\gl{Gleichung}{Quellgleichung des elektrischen Flusses}

\begingl
\begin{center}
	\formulaBegin
	$\Psi := \oiint \vec{D}\cdot d\vec{A} = \iint_V \rho dV = Q_{eff}$
	\formulaEnd

	Falls Feld senkrecht auf Fläche und konstant \\
	\fspace

	\formulaBegin
	$D = |\frac{Q_{eff}}{A_{eff}}|$
	\formulaEnd

\end{center}
\textbf{Variabeln}: \\
$\Psi = $ Elektrischer Fluss $ [\Psi] = A \cdot s $ \\
$D = $ Elektrische Flussdichte $ [D] = \frac{As}{m^2}$ \\
$ Q_{eff} = $ Von der Fläche eingeschlossene Ladung $[Q] = A\cdot s$ \\
$ A_{eff} = $ Fläche, durch die das D-Feld durchfliesst$ [A] = m^2$ \\

\iend

\bsptask{Beispiel}{Berechnung des D-Feldes einer Kugel}
\beginbsp
Eine Kugel sei mit der Ladung $20 C$ gefüllt. Die Kugel habe den Radius $R = 2cm$ und alle Ladungsträger verteilen sich auf der Kugeloberfläche. \\
Berechne das $\vec{D}$-Feld der Kugel.
\iend


\newpage


\important{Lösung}{}
\beginbsp
Um das D-Feld der Kugel anzugeben, müssen wir zuerst ein passendes Koordinatensystem wählen. \\
Da wir mit einer Kugel rechnen, muss das Feld punktsymmetrisch sein und einzig der Abstand zum Kugelmittelpunkt bestimmt, wie gross das D-Feld ist. \\
Aufgrund dieser Überlegung, werden wir \textbf{Kugelkoordinaten} verwenden. (Hätten wir etwas, das symmetrisch zur Z-Achse ist, so würden wir Zylinderkoord. verwenden etc.). \\
Um die Gleichung für den elektrischen Fluss anwenden zu können, müssen wir eine Hüllfläche finden,   \textbf{auf der das D-Feld konstant ist}. \\
Da die Anordnung eine Kugel ist, verwenden wir als Hüllfläche eine Kugel mit Radius r. \\
Nun können wir die Quellgleichung verwenden:
\begin{center}
	$ \oiint_A \vec{D}\cdot d\vec{A} = Q_{eff} \rightarrow D \cdot  4\pi r^2  = Q_{eff} $
\end{center}
Der Betrag des D-Feldes entspricht also gerade der von der Hullfläche eingeschlossenen Ladung geteilt durch die Fläche. \\
Für den Fall, dass wir die Hüllfläche grösser als die Kugel wählen, schliessen wir alle Ladungen ein: \\
$\mathbf{r > R}$ \\
\begin{center}
	$Q_{eff} = 20C$ \\
	$D = \frac{20C}{4\pi r^2}$ \\
	$\vec{D} = D \cdot \vec{e}_r$
\end{center}

Falls unsere Hüllfläche im Inneren der Kugel ist, so schliessen wir keine Ladung ein: \\
$\mathbf{r < R}$ \\
\begin{center}
	$Q_eff = 0C$ \\
	$\vec{D} = 0$
\end{center}

Somit gilt für das D-Feld einer Kugel mit Ladungen auf der Oberfläche: \\
\begin{center}

	$ \displaystyle
	\vec{D}(r) =
	\begin{cases}
		0                                       & r < R \\
		\frac{20 C}{4 \pi r^2} \cdot  \vec{e}_r & r > R \\
	\end{cases}$

\end{center}
\iend



\newpage

Da die elektrische Flussdichte nur von den Ladungsträgern abhängig ist, lässt sie sich realativ einfach berechnen und bildet meist den Grundstein für
das Berechnen von E-Feldern usw. \\
\texttt{"}Fliesst\texttt{"} eine elektrische Flussdichte durch ein Material mit Ladungsträgern, so wird diese abgeschwächt, da sich im Inneren des Materiales ein
elektrisches Feld entgegen dem von Aussen angelegtem Feld ausbildet. \\
\begin{center}
	\ibox{
		\includegraphics[scale=0.4]{img/d-e-feld.png}}
\end{center}

\gl{Gleichung}{Zusammenhang E-Feld und D-Feld}
\begingl
\begin{center}
	\formulaBegin
	$ \vec{E} = \frac{1}{\varepsilon_0 \cdot \varepsilon_r} \cdot \vec{D}$
	\formulaEnd
\end{center}
\textbf{Variabeln}: \\
$D = $ Elektrische Flussdichte $ [D] = \frac{As}{m^2}$ \\
$ E = $ Elektrisches Feld $[E] = \frac{V}{m}$ \\
$ \varepsilon_0 = $ Dielektrizitätskonstante $ [\varepsilon_0] = \frac{C}{V\cdot m}$ \\
$ \varepsilon_r = $ rel. Permitivität. Unterschiedlich im Material $ [\varepsilon_r] = [ ]$ \\

\iend


\newpage


\definition{Verhalten von Feldgrössen bei Materialübergängen}
\beginip
Fliesst eine elektrische Flussdichte durch eine geladene Fläche, so verändert sich die Normalkomponente des D-Feldes. \\
Die Tangentialkomponente des Feldes bleibt jedoch konstant. \\
\begin{center}
	\ibox{
		\includegraphics[scale=1.5]{img/dfield.png}}
\end{center}
\formulaBegin
$\vec{D}_1 = \vec{D}_{1t} + \vec{D}_{1n} \Rightarrow \vec{D}_2 = \vec{D}_{1t} + (1 + \sigma) \cdot \vec{D}_{1n}$
\formulaEnd
\iend

\definition{Spannung}

\beginip
Wir definieren die Spannung zwischen zwei Punkten als das Wegintegral über das elektrische Feld: \\
\formulaBegin
$ U_{AB} :=  \int_A^B \vec{E} \cdot d\vec{s} $
\formulaEnd
{[U]} = Volt {[V]}
\iend

\paragraph{Bemerkung zur Spannung}

\begin{itemize}

	\item	Da die Beziehung $\vec{F} =  q \cdot \vec{E} $ gilt und die Arbeit als $ W_{AB} = \int_A^B \vec{F} \cdot d\vec{s}$ definiert ist, können wir die Spannung zwischen zwei Punkten als \textbf{Mass der benötigten Arbeit} um ein Ladungsträger von A nach B zu bringen betrachten.  \\
	\item Die Spannung ist unabhängig des Weges.  $U_{AC} = U_{AB} + U_{BC}$
	      \\ $\Rightarrow$ Start- und Endwert sind ausreichend. \\
	\item Die Spannung über einen geschlossenen Weg entspricht \textbf{0V}: $U_{AB} + U_{BA} = 0$ \\
	\item Ist die Spannung auf dem gesamten Weg konstant und parallel zum Weg, vereinfacht sich das Integral zu einer Multiplikation: \\
	      \begin{center}
	      	$ U_{AB} = \int_{A}^{B} \vec{E}\cdot d\vec{s} = l_{AB} \cdot E$
	      \end{center}
\end{itemize}


\newpage
\subsection{Vorgehen zur Feldberechnung}
\vorgehen{Vorgehen}{Feldberechnungen für eine gegebene Anordnung}
\beginvor
Zuerst: Besteht die Anordnung aus bekannten Teilobjekten (Kugeln mit verschiedenen Radien etc?) \\
$\rightarrow$ Wende \textbf{Superposition} an: Berechne das Feld jeder einzelnen Teilanordnung mithilfe des Vorgehens und summiere Resultate auf.
\\
\begin{itemize}

	\item [1. ] Versuche die Symmetrie der Anordnung herauszufinden und entscheide dich für ein Koordinatensystem: \\
	      Punktsymmetrisch $\rightarrow$ Kugelkoordinaten \\
	      Achsensymmetrisch $\rightarrow$ Zylinderkoordinaten \\
	      Symmetrisch zu einer Ebene $\rightarrow$ karthesische koordinaten\\

	\item [2. a]Falls \textbf{Ladungsdichte oder Ladung} gegeben: \\
\end{itemize}
\beginip
\begin{itemize}

	\item [2. 1]  Suche Hüllfläche, auf der das Feld konstant ist


	\item [2. 2] Verwende $\displaystyle \oiint_A \vec{D} \cdot d\vec{A} = Q \Rightarrow D = \frac{Q_{eff}}{A_{eff}}$ \\
	      Je nach Hüllfläche muss hier eine Fallunterscheidung gemacht werden, da ggf. sich die eingeschlossene Ladung $Q_{eff}$ ändern könnte\\
	      Falls das D-Feld senkrecht durch verschiedene Materialien fliesst, bleibt es überall konstant.

	\item [2. 3] Aus Skizze, finde heraus, in welche Richtung das D-Feld zeigt und ergänze den Richtungsvektor:
	      \begin{center}
	      	$\vec{D} = D \cdot \vec{e}_D$
	      \end{center}

	\item [2. 4] Das resultierende E-Feld in den einzelnen Materialien mit $\varepsilon_{ri}$ ist gegben als:
	      \begin{center}
	      	$\displaystyle \vec{E} = \frac{\vec{D}}{\varepsilon_0 \cdot \varepsilon_{ri}}$
	      \end{center}
\end{itemize}
\iend
\begin{itemize}
	\item [2. b] Falls \textbf{Spannung} zwischen Punkten gegeben ist:\\

\end{itemize}
\beginip
\begin{itemize}

	\item [2. 1]  Suche eine Weg vom Punkt A zum Punkt B, auf dem das Feld konstant ist. \\

	\item [2. 2] Verwende $\displaystyle U_{AB} = \int_A^B \vec{E}\cdot d\vec{s}$ \\
	      $\displaystyle \rightarrow E = \frac{U_{AB}}{l_{AB}}$

	\item [2. 3] Aus Skizze, finde heraus, in welche Richtung das E-Feld zeigt und ergänze den Richtungsvektor:
	      \begin{center}
	      	$\vec{E} = E \cdot \vec{e}_E$
	      \end{center}
\end{itemize}
\iend
\iend


\newpage
\subsection{Kondensator}

Bringen wir 2 Platten mit verschiedenen Ladungsträger nah zu einander, so bildet sich ein elektrisches Feld zwischen den Platten. \\
Die Stärke des elektrischen Feldes ist abhängig der Plattenladung und der Flächen der Platten.

\fix
\fix
\fix
\definition{Kondensator}
\beginip
Ein Kondensator ist ein Bauelement, das in der Lage ist elektrische Ladung und somit Energie in Form von Feldlinien zu speichern. \\
Die charakteristische Kenngrösse des Kondensators ist die Kapazität \textbf{C}. \\
Die im Feld eines Kondensators gespeicherte Energie entspricht $\mathbf{W = \frac{1}{2} C \cdot U^2}$ \\
\iend



\fix
\fix
\fix
\definition{Kapazität}
\beginip
Die elektrische Kapazität C beschreibt die Fähigkeit eines Bauelementes, Ladung $Q$ bei einer gewissen Spannung $U$ zu speichern. \\
Als Proportionalitätsgrösse gibt sie an, wie gross die Spannung über einem Bauelement bei einer Ladung Q ist.
\formulaBegin
$C =\displaystyle \frac{Q}{U}$
\formulaEnd
In Abhängigkeit der Felder, lässt sich die Kapazität folgendermassen beschreiben:
\formulaBegin
$ C = \displaystyle \frac{\oiint_A \vec{D} \cdot d \vec{A} }{ \int_s \vec{E} \cdot d\vec{s}}$
\formulaEnd
Falls das Feld senkrecht auf der Hüllfläche steht und der Integrationsweg parallel zum E-Feld verläuft gilt:
\formulaBegin
$ C= \displaystyle \frac{D \cdot A_{eff} } {\int_A^B E \cdot ds} $
\formulaEnd
\iend


\gl{Gleichung}{Kapazität eines Plattenkondensators}
\begingl
Bei einem Plattenkondensator mit Fläche A und Abstand d, der mit einem Dielektrikum mit konstante $\varepsilon$ gefüllt ist, gilt: \\
\formulaBegin
$ \displaystyle C = \varepsilon \cdot \frac{A}{ d}$
\formulaEnd

\textbf{Variabeln}: \\
$A = $ Fläche einer Platte $ [A] = m^2 $ \\
$d = $ Abstand der Platten $ [d] = m$ \\
$ \varepsilon = \varepsilon_0 \cdot \varepsilon_r $ Dielektrikum zw. den Platten $ [\varepsilon] = \frac{C}{V\cdot m}$ \\

\iend

\textbf{Begründung}
Da das E-Feld eines Plattenkondensators konstant ist, gilt für die Kapazität:
\fix
\fix
\begin{center}
	$ C= \displaystyle \frac{D \cdot A_{eff} } {E \cdot d} $
\end{center}
\fix
\fix
Mit dem Zusammenhang $ D = \varepsilon \cdot E$ und der Erkenntniss, dass das Feld nur auf der einen Seite der Platte existiert folgt:
\fix \fix
\begin{center}
	$ C= \displaystyle \frac{E \cdot \varepsilon \cdot A_{eff} } {E  \cdot d}  = \doubleunderline{\displaystyle \varepsilon \frac{A} { d}}$
\end{center}



\definition{Serienschaltung von Kapazitäten}

\beginip
Werden mehrere Kapazitäten seriell miteinander verbunden, so addieren sich die Kehrwerte der Kapazität \\
\formulaBegin
$\displaystyle \frac{1}{C_{ges}} = \sum_{i=0}^n \frac{1}{C_i} \Bigg\rvert$
$\displaystyle C_{ges} = \frac{C_1 \cdot C_2}{C_1 + C_2} = (C_1 || C_2)$
\formulaEnd
\iend


\textbf{Begründung} \\
Die Definition der Kapazität ist genau gegensätzlich zu der des Widerstandes ($ R \propto \frac{l}{A}$ , $ C \propto \cdot \frac{A}{d} $)\\
Werden Kondensatoren in Serie geschaltet, so vergrößert sich der effektive Abstand der Platten, weshalb wir die Kehrwerte addieren müssen. \\



\definition{Parallelschaltung von Kapazitäten}

\beginip
Werden mehrere Kapazitäten parallel miteinander verbunden, so addieren sich die Kapazitäten \\
\formulaBegin
$\displaystyle C_{ges} = \sum_{i=0}^n C_i = C_1 + C_2 + ...$
\formulaEnd
\iend

\definition{Ladungserhaltung in der Serienschaltung}
\beginip
Werden Kapazitäten in Serie geschaltet, besitzen alle dieselbe Ladung Q.
\begin{center}
	\fix
	\ibox{\includegraphics[scale=0.5]{img/kond1.png}} $\displaystyle \doubleunderline{Q_1 = Q_2 = Q}$
\end{center}
\iend

\textbf{Begründung} \\
Damit sich auf dem ersten Kondensator die Ladung $Q_1$ ansammeln kann, muss diese Ladung unterhalb des Kondensators angesammelt werden. Angenommen, beide Kondensatoren waren zu Beginn ungeladen,
so muss die Ladung, welche sich auf der unteren Platte des ersten Kondensators befindet, dieselbe Ladungsmenge auf der oberen Platte des unteren Kondensators hervorrufen. \\
\fix
\fix
\fix
\definition{Maschenregel bei Parallelschaltung}
\beginip
Die Maschenregel für Spannung gilt auch bei Kondensatoren
\begin{center}
	\fix
	\ibox{\includegraphics[scale=0.5]{img/kond2.png}}
	$\displaystyle \doubleunderline{U_1 = U_2 = U} $  und  $\displaystyle C_1 = \frac{Q_1}{U_1} \rightarrow \doubleunderline{Q_1 = C_1 \cdot U}$
\end{center}
\iend


	 \subsection{Strom}
\label{chap:Style}
Legen wir ein elektrisches Feld an einem Material mit beweglichen Ladungsträgern an,
so beginnen sich die Ladungsträger entlang dem angelegten Feld zu bewegen. \\
Je mehr freie Ladungsträger pro Volumen vorhanden sind ($:= \rho$), desto mehr Ladungsträger werden sich zu bewegen beginnen. \\
Die Geschwindigkeit ($:= v,$ Driftgeschwindigkeit), mit der sich die Ladungsträger bewegen, ist abhängig von der Stärke des elektrischen Feldes und
dem Material selbst. Wie gut sich ein Ladungsträger in einem Material bewegen kann, beschreiben wir mit der Beweglichkeit ($:= \mu$)


\important{Definition} {Strom und Stromdichte}
\beginip
Der Strom I bezeichnet, wie viele Teilchen sich pro Zeit durch eine Fläche bewegen. \\
Die Stromdichte J sagt etwas darüber aus, wie viel Strom pro Fläche fliesst \textbf{(= Dichte)}
\formulaBegin
$\displaystyle I := \frac{dQ}{dt} = \iint_A \vec{J} \cdot d\vec{A}$
\formulaEnd
\textbf{Variabeln}: \\
$ I = $ Strom $ [I] = A = \frac{C}{s}$ \\

\formulaBegin
$\displaystyle \vec{J} = \underbrace{n\cdot q}_{\rho} \vec{v} = \kappa \cdot \vec{E}$
\formulaEnd

\textbf{Variabeln}: \\
$ \vec{J} = $ Stromdichte $ [J] = \frac{A}{m^2}$ \\
$ n =$ Teilchendichte $ [n] = \frac{1}{m^3}$ \\
$ q =$ Ladungen der Teilchen$ [q] = C = As$ \\
$ \vec{v} = $ Driftgeschwindigkeit $ [v] = \frac{m}{s}$ \\
$ \rho = $ Raumladungsdichte $ [\rho] = \frac{As}{m^3}$ \\
$ \kappa = $ Elektrische Leitfähigkeit $ [\kappa] = \frac{A}{V\cdot m}$
\iend

Grundlage für den Strom sind bewegte Ladungsträger. \\
Im Falle eines Kupferkabel sind dies zum Beispiel Elektronen, die sich \textbf{gegen} das
elektrische Feld bewegen. Da Elektronen jedoch eine negative Ladung besitzen und der Strom als Ladung pro Zeit die durch eine Fläche hindurchfliesst definiert ist, zählen negative Ladungen, die entgegen dem Strom fliessen, zum Strom hinzu.

\begin{center}
	$\displaystyle J := n \cdot q \cdot \vec{v_I} = n \cdot (-e) \cdot (-\vec{v_I}) $
\end{center}

In anderen Baustoffen (wie zum Beispiel Halbleitern), bewegen sich \textbf{positive Ladungsträger} (=Löcher) mit dem Elektrischen Feld. \\
Diese führen zu einem positiven Strom in dessen Bewegungsrichtung. \\
Je nach Material, ist es auch möglich, dass beide Ladungsträger zum Stromfluss beitragen, sich jedoch nicht gleich gut im Material bewegen können. \\
Aus diesem Grund, gibt es für positive wie negative Ladungen verschiedene Beweglichkeiten.

\definition{Elektrische Leitfähigkeit}
\beginip
Die elektrische Leitfähigkeit ($\kappa$) beschreibt, wie gross die Stromdichte in einem Material, bei einem gegebenen E-Feld wird. \\
\formulaBegin
$\displaystyle \vec{J} = \kappa \cdot \vec{E} = \vec{J_{-}} + \vec{J_{+}} = \underbrace{(n_{-} \cdot q_{-} \cdot \mu_- + n_+ \cdot q_+ \cdot \mu_{+})}_{\kappa} \cdot \vec{E}$
\formulaEnd
Dabei bezeichnen die Variabeln $\mu_{x}$ die \textbf{Beweglichkeit} der einzelnen Ladungsträger und sind ein Mass dafür, wie schnell sich die Teilchen bei einem gegebenem E-Feld bewegen werden.
\iend


\subsection{Verhalten des J- und E-Feldes an Materialübergängen}
Triftt eine Stromdichte auf einen Materialübergang, so ändert sich der Betrag der Tangentialkomponente. Die \textbf{Normalkomponente bleibt gleich}.
\begingl
Es gilt bei Materialübergängen: \\
Für das J-Feld:
\fspace
\formulaBegin

$\displaystyle \frac{tan(\alpha_1)}{tan(\alpha_2)} = \frac{J_{t1}}{J_{t2}}  $ \\
\fspace
$\displaystyle J_{n1} = J_{n2}$

\formulaEnd

Für das E-Feld:
\fspace
\formulaBegin

$\displaystyle E_{t1} = E_{t2}$
$\displaystyle \frac{tan(\alpha_1)}{tan(\alpha_2)} = \frac{E_{n2}}{E_{n1}}  $ \\
\fspace

\formulaEnd
\iend

	 \newpage
	 \section{Netzwerke}
\label{chap:Style}


\definition{Widerstand und Leitwert}

\beginip
Der \textbf{Widerstand} bestimmt, wieviel Strom fließen kann, wenn eine bestimmte Spannung angelegt wird. \\
\begin{center}
	\includegraphics[scale=0.25]{img/widerstand.png}
\end{center}
\formulaBegin
$ R :=  \frac{U}{I} =  \rho  \frac{l}{A}, \ \ \ \ \ \ \  {[R]} = \Omega, Ohm $
\formulaEnd
Als \textbf{Leitwert} bezeichnen wir die Inverse des Widerstandes. Er gibt an, wie groß die Spannung ist, wenn ein gewisser Strom fließt. \\
\formulaBegin
$ Y = \frac{1}{R} = \frac{A}{\rho \cdot l} $
\formulaEnd
\iend




Da das elektrische Feld wirbelfrei ist, erhalten wir unabhängig vom Weg den gleichen Wert für die Spannung $ U_{AB} $ \\
Dies bedeutet jedoch auch, dass wir für einen geschlossenen Weg die Spannung $0V$ erhalten müssen, da für jede geschlossene Kurve $\gamma$ gilt:
\begin{center}
	\vspace{-2mm}

	$\displaystyle \oint_{\gamma} \vec{E} \cdot d\vec{s} = \int_{\gamma_0}^{\gamma_1} \vec{E} \cdot d\vec{s} + \int_{\gamma_1}^{\gamma_0} \vec{E} \cdot d\vec{s} = U_{01} + U_{10} = 0$
\end{center}
Mit dieser Erkenntnis können wir die Maschenregel definieren:

\definition{Maschenregel}
\beginip
Die Summe aller Spannungen in einer Masche ergibt $0$ \\
\formulaBegin
$\displaystyle \sum_{k=1}^n U_k = 0$
\formulaEnd

\iend

\begin{minipage}{0.6\textwidth}
	\begin{flushright}
		\includegraphics[scale=0.45]{img/maschenregel-2.png}
	\end{flushright}
\end{minipage}
\begin{minipage}{0.4\textwidth}
	\textbf{Maschenregel} \\ \\
	I: $\displaystyle (- U_1) + U_3 + (-U_2)  = 0$ \\
	II: $\displaystyle U_3 + (- U_4) = 0 $ \\
	% \vspace{1.5cm}
\end{minipage}

Analog können wir mithilfe der Ladungserhaltung argumentieren, dass sämtliche Ladungen, welche in ein Gebiet hineinfließen, auch wieder aus diesem hinausließen müssen.

\definition{Knotenregel}
\beginip
Die Summe aller Ströme die in einen Knoten hinein/hinausfließen muss $0$ ergeben. \\
\formulaBegin
$\displaystyle\sum_{i=0}^n I_n = 0 $
\formulaEnd
\iend

\textbf{Wichtig} Die Knotenregel kann auch auf ein Gebiet von Knoten angewandt werden. \\


\begin{minipage}{0.6\textwidth}
	\begin{flushright}
		\includegraphics[scale=0.4]{img/knotengl.png}
	\end{flushright}
\end{minipage}
\begin{minipage}{0.4\textwidth}

	\textbf{Knotengleichungen} \\ \\
	$\displaystyle K_1$: $ I_1 - I_2 - I_3 = 0 $ \\
	$\displaystyle K_{45}$: $ I_2 + I_4 + I_5 - I_6 = 0 $ \\
\end{minipage}


\newpage

\subsection{Grundlegende Netzwerkumformungen}
Wir interessieren uns nun dafür, wie sich Widerstände verhalten, wenn wir sie seriell/parallel verknüpfen.
\definition{Serienschaltung}
\beginip
Werden mehrere Widerstände seriell miteinander verbunden, so addieren sich die Widerstandswerte \\
\formulaBegin
$\displaystyle R_{serie} = \sum_{i=0}^n R_i = R_1 + R_2 + ...$
\formulaEnd
\iend

\vspace{1em}

\textbf{Begründung} \\
Mehrere Widerstände in Serie können als einen langen Widerstand mit konstanter Fläche angesehen werden. Da die Längenabhängigkeit des Widerstandes im Zähler steht, addieren sich die Werte. \\
$\displaystyle R_s = \rho \cdot \frac{l_1+l_2}{A} = \rho \cdot \frac{l_1}{A}  + \rho \cdot \frac{l_2}{A}  = R_1 + R_2 $
\fix
\begin{center}
	\ibox{\includegraphics[scale=0.3]{img/serienschaltung.png}}
\end{center}
\fix


\definition{Parallelschaltung}

\beginip
Werden mehrere Widerstände parallel miteinander verbunden, so addieren sich die Leitwerte \\
\formulaBegin
$\displaystyle Y_{parallel} = \sum_{i=0}^n Y_i \Bigg\rvert$
$\displaystyle R_{parallel} = \left(\sum_{i=0}^n \frac{1}{R_i}\right)^{-1} = \frac{(R_1 \cdot R_2)}{R_1 + R_2}$
\formulaEnd
\iend

\vspace{1em}

\textbf{Begründung} \\
Mehrere Widerstände parallel können als einen Widerstand mit größerer Fläche und konstanter Länge angesehen werden. Da die Flächenabhängigkeit des Widerstandes im Nenner steht, addieren sich die Leitwerte. \\
$\displaystyle Y_p = \frac{1}{\rho} \cdot \frac{A_1 + A_2}{l} = \frac{1}{\rho} \cdot \frac{A_1}{l}  + \frac{1}{\rho} \cdot \frac{A_2}{l}  = Y_1 + Y_2 $


\begin{center}
	\ibox{\includegraphics[scale=0.3]{img/parallel.png}}
\end{center}




\subsection{Stern Dreieck Umformung}
Sind Widerstände weder seriell noch parallel geschaltet, sondern sind in einem Stern oder Dreieck angeordnet,
können folgende Zusammenhänge verwendet werden, um eine Dreieckschaltung in eine Sternschaltung und umgekehrt umzuwandeln.
\begin{center}
	\includegraphics[scale=0.3]{img/Stern-Dreieck-Transformation.png}
\end{center}

\begin{minipage}{0.5\textwidth}
	\begin{center}

		${R}_{AB}=\frac{{R}_{A}{R}_{B}+{R}_{B}{R}_{C}+{R}_{A}{R}_{C}}{{R}_{C}}$\\
		${R}_{AC}=\frac{{R}_{A}{R}_{B}+{R}_{B}{R}_{C}+{R}_{A}{R}_{C}}{{R}_{B}}$\\
		${R}_{BC}=\frac{{R}_{A}{R}_{B}+{R}_{B}{R}_{C}+{R}_{A}{R}_{C}}{{R}_{A}}$\\

	\end{center}
\end{minipage}
\begin{minipage}{0.5\textwidth}
	\begin{center}

		${R}_A=\frac{{R}_{AC}{R}_{AB}}{{R}_{AC}+{R}_{AB}+{R}_{BC}}$\\
		${R}_B=\frac{{R}_{AB}{R}_{BC}}{{R}_{AC}+{R}_{AB}+{R}_{BC}}$\\
		${R}_C=\frac{{R}_{AC}{R}_{BC}}{{R}_{AC}+{R}_{AB}+{R}_{BC}}$

	\end{center}
\end{minipage}

\newpage



\gl{Gleichung}{Spannungsteiler}
\begingl
Die Spannungsteilerregel gibt an, wie sich eine Spannung über verschiedene Widerstände aufteilt, wenn diese in \textbf{Serie} geschaltet sind. \\
\formulaBegin
$\displaystyle U_{R_x} = U_{ges} \cdot \frac{R_X}{\sum R_i}$
\\
2 Widerstände: $
U_{R_1}  = U_{ges} \cdot \frac{R_1}{R_1 + R_2}$
\formulaEnd

\begin{center}
	\ibox{\includegraphics[scale=0.4]{img/spannungsteiler.png}}
\end{center}
\iend


\textbf{Begründung} \\
Gemäss $\displaystyle U = R \cdot I $ und der Serienschaltung ist der Strom durch alle Widerstände gegeben als $\displaystyle I = \frac{U_{ges}}{\sum R_i} $ \\
Nun müssen wir nur noch den Strom mit dem gesuchten Widerstand multiplizieren um die Spannung zu erhalten: $\displaystyle U_{R_X} = R_X \cdot I = R_X \frac{U_{ges}}{\sum R_i} $

\fix
\gl{Gleichung}{Stromteiler}
\begingl
Die Stromteilerregel gibt uns an, wie sich die Ströme in einem Knoten aufteilen, wenn die Widerstände \textbf{parallel} geschaltet sind.
\fspace
\formulaBegin
$\displaystyle I_x = I_{in} \cdot \frac{(R_1 || ... || R_n)} {R_x} $
\formulaEnd
\fix
\begin{center}
	\ibox{\includegraphics[scale=0.3]{img/stromteiler.png}}
\end{center}
\fix
\textbf{Spezialfall} 2 Widerstände \\
Falls der Stromteiler nur mit zwei Widerständen angewendet wird, vereinfacht sich die Formel:
\fspace
\formulaBegin
$\displaystyle I_x = I_{in} \cdot \frac{R_y}{R_x + R_y} $
\formulaEnd
Der Widerstand, dessen Strom uns \textbf{nicht} interessiert, steht hierbei im Zähler!
\iend

\subsection{Grundregeln bei Netzwerkumformungen}
\fix \fix
\important{Regel 1}{Expandieren von Knoten}
\beginip
Knoten können aufgeteilt und mit Verbindungslinien verbunden werden\\
\begin{center}
	\ibox{\includegraphics[scale=0.25]{img/knotenexp.png}}
\end{center}
\iend

\fix
\fix \fix
\important{Regel 2}{Verschieben von Elementen}
\beginip
Elemente können entlang von \textbf{Verbindungslinien ohne Widerständen} verschoben werden\\
\begin{center}
	\fix \fix \fix
	\ibox{\includegraphics[scale=0.3]{img/verschieben.png}}
\end{center}
\iend
\fix \fix
\important{Regel 3}{Vertauschen von Elementen in Serie}
\beginip
Elemente, die \textbf{in Serie} geschaltet sind, können vertauscht werden
\begin{center}
	\fix
	\ibox{\includegraphics[scale=0.3]{img/vertausch-serie.png}}
\end{center}
\iend
\fix \fix
\important{Regel 4}{Vertauschen von parallel geschalteten Elementen}
\beginip
Elemente die \textbf{parallel} geschaltet sind, können vertauscht werden
\begin{center}
	\fix
	\ibox{\includegraphics[scale=0.3]{img/vertausch-parallel.png}}
\end{center}
\iend
\fix \fix
\important{Regel 5}{Knoten kontrahieren}
\beginip
(Analog zu 1) Knoten können zusammengezogen werden, sofern sie nicht durch einen Widerstand verbunden sind.
\begin{center}

	\ibox{\includegraphics[scale=0.3]{img/knoten-kontrahieren.png}}
\end{center}
\iend

\newpage


\subsection{Vereinfachungen mithilfe Symmetrieüberlegungen}

Besitzt ein Netzwerk verschiedene Punkte mit dem selben Potential, so können diese Punkte beliebig verbunden werden.
Da die Potentialdifferenz immer 0 sein wird, wird niemals Strom zwischen diesen Punkten fliessen.

\bsptask{Beispiel}{Vereinfachung eines Netzwerkes mit Symmetrie}
\beginbsp
Fassen sie alle Widerstände zu einem zusammen unter Verwendung der Symmetrieeigenschaften
\begin{center}
	%TODO add ppoints
	\ibox{ \includegraphics[scale=1.5]{img/sym.png}}
\end{center}
\iend

\bsp{Lösung}{}
\beginbsp
Da das Widerstandsnetzwerk symmetrisch bezüglich $R_2$ ist, muss am Punkt 2 und am Punkt 4 das gleiche Potential existieren. \\
Dies bedeutet, dass über dem Widerstand $R_2$ niemals eine Spannug abfallen und somit auch nie Strom fliessen wird. \\
\textbf{Lösung 1}: \\
Wir verbinden die Punkte 2 und 4 mit einem Kurzschluss und erhalten
\begin{center}
	\ibox{ \includegraphics[scale=1]{img/sym-l1.png}}
	$\doubleunderline{R_{ges}} = (R_1 || R_1) + (R_3 || R_3) = \doubleunderline{\frac{1}{2}(R_1 + R_3)}$
\end{center}

\textbf{Lösung 2}: \\
Wir verbinden die Punkte 2 und 4 mit einem Leerlauf und erhalte
\begin{center}
	\ibox{ \includegraphics[scale=1]{img/sym-l2.png}}
	$\doubleunderline{R_{ges}} = \big( (R_1 + R_3) || (R_1 + R_3) \big) = \doubleunderline{\frac{1}{2}(R_1 + R_3)}$
\end{center}


\iend


\newpage

\subsection{Vorgehen um Schaltbilder mit einer Quelle zu vereinfachen}
\begin{enumerate}
	\item Bringe Quelle auf die linke Seite
	\item Forme mit Regel 1 - 5 das Netzwerk soweit um, bis nur noch Spannungs-/Stromteiler oder einfache Maschen vorhanden sind.
	\item Expandiere nun das Netzwerk Schritt für Schritt, bis die Spannung über dem gesuchten Widerstand berechnet werden kann.
\end{enumerate}

\bsptask{Beispiel}{}
\beginbsp
1) Berechnen Sie $U_x$ in Abhängigkeit des Quellstromes $I$
\begin{center}
	\fix
	\ibox{\includegraphics[scale=0.55]{img/aufg1-aufg.png}}
\end{center}
\fix
\iend
\bsp{Lösung}{}
\beginbsp
\begin{enumerate}
	\item Gemäss Regel 2 können wir $R_1$ und die Stromquelle vertauschen.
	\item Die Widerstände $R_4 , R_5$ und $ R_x $ können zu $ R_S = R_X + R_4 + R_5 $ zusammengefasst werden.
	\item Mithilfe der Stromteilerregel erhalten wir $ I_s = I \cdot \frac{(R_1 || R_2 || R_s)}{R_s} $ und somit $U_s = I \cdot (R_1 || R_2 || R_S) $
	\item Die Spannung $U_s$ liegt also über dem Widerstand $R_s$ an. Wenn wir diesen wieder in die ursprünglichen 3 Widerstände aufteilen, erhalten wir mithilfe der Spannungsteilerregel $U_x = U_s \cdot \frac{R_x}{R_4 + R_5 + R_x} $
\end{enumerate}

\ibox{\includegraphics[scale=0.55]{img/bsp1.png}}
\iend

In manchen Fällen kann es schwierig sein, die Quelle auf die linke Seite zu bringen oder das Schaltbild nützlich umzuformen. \\
Um ein erstes Ersatzschaltbild zu erhalten, kann das "Flussverfahren\texttt{"} angewandt werden.

\newpage
\bsp{Beispiel}{Flussverfahren}
\beginbsp
Beim Flussverfahren überlegt man sich sämtliche Arten, wie der Strom von einem Ende der Quelle zum anderen fliessen kann, und zeichnet somit ein Ersatzschaltbild. \\ \\
\textbf{Beispiel}
\begin{center}
	\fix
	\ibox{\includegraphics[scale=0.3]{img/fluesse.png}}
\end{center}
\iend

\subsection{Quellen}
\definition{Ideale Quelle}
\beginip
Eine ideale Strom-/Spannungsquelle liefert immer denselben Strom/dieselbe Spannung, unabhängig von der Last, welche angehängt wird.
\begin{center}
	\ibox{\includegraphics[scale=0.6]{img/ideale_quelle.png}}
\end{center}

\iend


Mit idealen Strom-/Spannungsquellen können wir theoretisch unendlich viel Spannung/Strom über einem Lastwiderstand erzeugen. \\
(Beispiel ideale Spannungsquelle im Kurzschluss/ideale Stromquelle im Leerlauf) \\
Bei einer realen Quelle kann jedoch nur eine endliche Spannung/Strom auftreten, wesshalb wir Verluste innerhalb der Quelle mit einem Innenwiderstand $R_i$ modellieren. \\

\definition{Reale Quelle}
\beginip
Eine reale Quelle bezeichnet eine ideale Quelle mit Vorwiderstand. \\
Bei einer \textbf{Stromquelle} ist der Widerstand \textbf{parallel}, bei einer \textbf{Spannungsquelle} ist der Widerstand in \textbf{Serie}.
\begin{center}
	\fix
	\ibox{\includegraphics[scale=0.6]{img/realeQuellen.png}}
\end{center}
\iend




\subsection{Superpositionsprinzip}

Das Superpositionsprinzip besagt, dass wir bei einem Netzwerk mit mehreren Quellen einzelne Teillösungen in Abhängigkeit von nur einer Quelle berechnen und aufsummieren können. \\
Dies gilt jedoch nicht für die Leistung, da diese nicht linear ist! \\
Um das Superpositionsprinzip anzuwenden, müssen wir alle ausser einer Quelle auf \texttt{"}0\texttt{"} setzen. \\
Spannungsquellen werden also mit \textbf{Kurzschlüssen} ersetzt und Stromquellen mit \textbf{Leerläufen}. \\
\begin{center}
	\fix
	\ibox{\includegraphics[scale=0.3]{img/superpos-zero.png}}
\end{center}


\textbf{Wieso?} \\
Eine Spannungsquelle zu \texttt{"}0\texttt{"} zu setzen bedeutet, dass über diesem Bauteil keine Spannung abfallen darf. Über einem Kurzschluss wird nie eine Spannung abfallen, da dieser als Widerstand mit Wert 0 modelliert werden kann. \\
Eine Stromquelle zu \texttt{"}0\texttt{"} zu setzen bedeutet, dass durch dieses Bauteil kein Strom fliessen darf. Dies entspricht gerade einem Leerlauf, da dieser als Widerstand mit Wert $\displaystyle \rightarrow \infty$ modelliert werden kann. \\

\newpage


\bsptask{Beispiel}{}
\beginbsp
Berechnen Sie die Spannung $U_x$ und die Leistung $P_x$ im folgenden Netzwerk, wenn alle Widerstände $ R = 100 \Omega$ betragen. \\
\begin{center}
	\fix
	\ibox{\includegraphics[scale=0.6 ]{img/ex2-1.png}}
\end{center}
\iend
\newpage
\bsp{Lösung}{}
\beginbsp
Zuerst setzen wir die Spannungsquelle zu 0 und erhalten das folgende ESB \\
\begin{center}
	\fix
	\ibox{\includegraphics[scale=0.5 ]{img/ex2-2.png}}
\end{center}

Nun berechnen wir den Strom $I_{x_1}$ durch den Widerstand mithilfe eines Stromteilers: \\
$I_{x_1} = 60mA$ \\
Die Spannung $U_x$ ist entgegen der Stromrichtung eingezeichnet: \\
$U_{x_1} = - R_x \cdot I_x = - 100 \cdot 60mA = -6V $ \\
\\
Nun setzen wir die Stromquelle zu 0: \\
\begin{center}
	\fix
	\ibox{\includegraphics[scale=0.5 ]{img/ex2-3.png}}
\end{center}

Die Spannung $U_{x_2}$ berechnet sich als Spannungsteiler: \\
$U_{x_2} = 20V \cdot \frac{100\Omega}{400\Omega} = 5V$ , $ I_{x_2} = \frac{5V}{100\Omega} = 50mA $\\
\\
\\
\textbf{Superposition:} \\
Schlussendlich berechnet sich die Spannung als Summe der Teilspannungen: \\
$\displaystyle U_x = U_{x_1} + U_{x_2} = 5V -6V = -1V $ \\
Und die Leistung: \\
$\displaystyle P_x = \frac{{U_x}^2}{R_x} = 10mW$ \\
Welche \textbf{nicht} der Summe der Teilleistungen entspricht: \\
$\displaystyle P_{sum} = P_1 + P_2 = U_{x_1} \cdot I_{x_1} + U_{x_2} \cdot I_{x_2} = 360mW + 250mW = 610mW$


\iend

\newpage

\subsection{Ersatzquelllen}


\definition{Thévenin / Norton Äquivalent}
\beginip
Jedes Netzwerk mit \textbf{linearen} Bauelementen und 2 Klemmen lässt sich als reale Quelle darstellen. \\
\textbf{Thévenin Äquivalent} Darstellung als reale \textbf{Spannungsquelle} mit Leerlaufspannung, die an den Klemmen auftritt \\
\textbf{Norton Äquivalent} Darstellung als reale \textbf{Stromquelle}  mit Kurzschlussstrom, der an den Klemmen auftritt\\
Der Innenwiderstand entspricht dem von außen gemessenen Widerstand, wenn alle Quellen zu 0 gesetzt werden.
\begin{center}
	\fix
	\ibox{\includegraphics[scale=0.6]{img/nort_the.png}}
\end{center}
\iend


\newpage
\vorgehen{Vorgehen}{Ersatzquelle für gegebenes Schaltbild mit offenen Klemmen finden}
\beginvor
\begin{itemize}

	\item[1.] Betrachte die Schaltung. Sind mehr Stromquellen vorhanden, so ist es häufig einfacher den Kurzschlussstrom zu berechnen. Bei mehr Spannungsquellen die Leerlaufsspannung. \\
	      Wiederhole folgendes für alle Quellen $i$: \\
\end{itemize}
\beginip
\begin{itemize}

	\item [2.  ]  Setze alle Strom / Spannungsquellen ausser einer zu 0. (Stromquelle $\rightarrow$ Leerlauf, Spannungsquelle $\rightarrow$ Kurzschluss)
	\item[3. ] Versuche die einzelne Quelle auf die linke Seite zu bekommen. (Siehe Skript \texttt{"}Flussverfahren\texttt{"})
	\item[4.a)] \textbf{Kurzschlussstrom}
	      \begin{enumerate}
	      	\item Schliesse die Klemmen kurz und bezeichne den Strom, welcher durch diese Klemmen fliesst als $I_{ks}^{(i)}$.
	      	      %TODO ADD BILDER
	      	\item Versuche mittels Stromteilern den gesuchten Strom zu berechnen. \textit{Falls eine direkte Verbindung von Stromquelle über den Kurzschluss zur Quelle zurückfürt, ist der Kurzschlussstrom gleich dem Strom der Quelle. (Bild)}\\
	      	      Ist die Quelle keine Stromquelle, so kann evt. ein serieller Widerstand verwendet werden, um die Quelle umzuformen.
	      \end{enumerate}
	\item[4.b)] \textbf{Leerlaufspannung}
	      \begin{enumerate}
	      	\item Zeichne einen Spannungspfeil zwischen den Klemmen und bezeichne die Spannung als $U_{LL}^{(i)}$.
	      	\item Versuche mittels Spannungsteiler die gesuchte Spannung zu berechnen. \textit{Falls kein Strom von der Spannungsquelle fliessen kann (Leerlauf unterbricht die komplette Schaltung), so ist die Leerlaufspannung gleich der Spannung der Spannungsquelle. (Bild)} \\
	      	      Ist die Quelle keine Spannungsquelle, so kann evt. ein paralleler Widerstand verwendet werden, um die Quelle umzuformen.
	      \end{enumerate}
\end{itemize}

\iend
\begin{itemize}

	\item[5.] \textbf{Innenwiderstand}
	      \begin{enumerate}
	      	\item Setze \textbf{alle} Quellen auf 0. Versuche nun die Widerstände solange umzuformen, bis nur noch ein Ersatzwiderstand vorhanden ist. (Ggf. \texttt{"}Flussverfahren\texttt{"} mit offener Klemme anwenden)
	      	\item Bezeichne den Wert des Widerstandes als $R_i$
	      \end{enumerate}
	\item[6.a)] \textbf{Thévenin Äquivalent} Spannungsquelle mit seriellem Innenwiderstand. \\
	      Werte: $\displaystyle   R= R_i$, $U_q = \sum_i U_{LL}^{(i)} = (\sum_i I_{ks}^{(i)})\cdot R_i$

	\item[6.b)] \textbf{Norton Äquivalent} Stromquelle mit parallelem Innenwiderstand. \\
	      Werte: $\displaystyle  R= R_i$, $I_q = \sum_i I_{ks}^{(i)}) = \frac{\sum_i U_{LL}^{(i)}}{R_i}$

\end{itemize}

\iend
\begin{center}


	\begin{minipage}{0.4\textwidth}
		\begin{center}
			\includegraphics[scale=0.7]{img/the_ks.PNG} \\
			Bild zu 4.a)
		\end{center}



	\end{minipage}
	\begin{minipage}{0.4\textwidth}
		\begin{center}
			\includegraphics[scale=0.7]{img/the_ll.PNG}\\
			Bild zu 4.b)
		\end{center}
	\end{minipage}

\end{center}





\newpage
\bsptask{Beispiel}{Thévenin Äquivalente Schaltung}
\beginbsp
Geben Sie eine Thévenin Äquivalente Schaltung (Innenwiderstand $R_i$, $U_{Th}$) für folgende Klemmen an. Alle Widerstände haben Wert $100\Omega$ \\

\begin{center}
	\fix
	\ibox{\includegraphics[scale=0.4]{img/ex3-1.png}}
\end{center}
\iend
\bsp{Lösung}{}
\beginbsp																				Zuerst berechnen wir die Leerlaufspannung für die Stromquelle: \\
$U_{LL_1} = 120mA \cdot 200 \Omega = 24V$ \\
Danach die Leerlaufspannung für die Spannungsquelle: \\
$ U_{LL_2} = 20V$\\
Die Gesamtspannung und somit die Spannung der Ersatzspannungsquelle beträgt: \\
$\doubleunderline{U_{Th}} = U_{LL_1} + U_{LL_2} = \doubleunderline{44V}$ \\

Nun müssen wir noch den Innenwiderstand berechnen. Dazu setzen wir alle Quellen zu 0 und berechnen den von aussen gemessenen Widerstand:\\
\begin{center}
	\fix
	\ibox{\includegraphics[scale=0.6]{img/ex3-2.png}}
\end{center}
$\doubleunderline{R_i} = 100\Omega + 100\Omega  + 100\Omega  = \doubleunderline{300 \Omega} $
\iend

\newpage
\subsection{Leistungsanpassung}
\definition{Leistung}
\beginip
Als Leistung bezeichnen wir das Produkt von Strom und Spannung. \\
Sie bezeichnet die in einer Zeitspanne umgesetzte Energie an einem Bauteil. \\
\formulaBegin
$\displaystyle P := U \cdot I = \frac{U^2}{R} = I^2 \cdot R $
\formulaEnd
\iend

\definition{Maximale Leistung}
\beginip
Um bei einer realen Quelle maximale Leistung an einen Lastwiderstand abzugeben, muss der Lastwiderstand gleich gross sein wie der Innenwiderstand der Quelle. \\
\formulaBegin
$ R_i = R_L \Rightarrow P = P_{max}$
\formulaEnd
\begin{center}
	\ibox{\includegraphics[scale=0.5]{img/leistungsanpassung.png}}
\end{center}
\iend
\textbf{Begründung} \\
Die Leistung an einem Lastwiderstand in Serie ist gegeben als: \\
$ \displaystyle P_L = U_L \cdot I_L = \frac{U_L^2}{R_L} = \frac{( U_0 \frac{R_L}{R_L + R_i})^2}{R_L} = U_o^2 \cdot \frac{R_L}{(R_L + R_i)^2}$ \\
$\displaystyle \frac{d}{dR_L} (P_L) = - U_0^2 \cdot \frac{R_L - R_i}{(R + R_L)^3} = 0 \Rightarrow R_L = R_i  \Rightarrow P_L = P_{max}$





















\vorgehen{Vorgehen}{Leistung über einer Last maximieren}
\beginvor
\textbf{Falls möglich}
\begin{itemize}
	\item [1. ] Entferne Komponenten über denen die Leistung maximiert werden soll und ersetze sie mit offenen Klemmen (Fasse die entfernten Komponenten als einen Lastwiderstand zusammen).
	\item [2. ] Berechne von den Klemmen den Innenwiderstand.
	\item [3. ] Innenwiderstand $R_i \rightarrow R_L = R_i$
	\item [4. ] Falls maximale Leistung gefragt: Berechne Leerlaufspannung oder den Kurzschlussstrom. $\displaystyle P_{max} = \frac{U_{LL}^2} {4 \cdot R_i } $
\end{itemize}

\textbf{Sonst}

\begin{itemize}
	\item [1. ] Finde einen Ausdruck für $U_L$ und $I_L$
	\item [2. ] Berechne $P = U_L \cdot I_L$
	\item [3. ] Leite $P$ nach dem veränderlichem Widerstand ab und setze zu 0 um ein Maximum zu finden
\end{itemize}


\iend
\newpage
\bsptask{Beispiel}{Leistungsanpassung}
\beginbsp
Berechnen sie für folgendes ESB den Wert für $R_L$ bei dem $R_L$ maximale Leistung aufnimmt.
\begin{center}
	\ibox{\includegraphics[scale=0.2]{img/leistungsanp.png}}
\end{center}
\iend

\bsp{Lösung}{}
\beginbsp
Um die Leistung über $R_L$ zu maximieren, entfernen wir den Widerstand und berechnen den Innenwiderstand bezüglich den Klemmen. \\
Für die Berechnung des Innenwiderstandes, werden alle Spannungsquellen kurzgeschlossen.
\begin{center}
	\ibox{\includegraphics[scale=0.6]{img/leistungsanp_lsg.png}}
\end{center}
Nun müssen nur noch alle Widerstände zu einem zusammengefügt werden:
\begin{center}
	$R_i = (R_1 || (R_2 + (R_3 || R_4 ) )) = 2 \Omega$ \\
	$\doubleunderline{ R_L = R_i = 2 \Omega}$
\end{center}
\iend










\newpage





\subsection{Analyse umfangreicher Netzwerke}

Möchten wir in Netzwerken sehr viele Grössen berechnen, oder ist das Netzwerk sehr komplex, so können andere (meist Computer basierende) Verfahren verwendet. \\
Sämtliche Verfahren bauen darauf auf, dass für ein beliebiges Netzwerk das Aufstellen sämtlicher Knoten und Maschengleichungen genügt, um alle gesuchten Grössen zu berechnen. \\

\important{Unabhängige Gleichungen}{}
\beginip
Seien z die Anzahl Zweige eines Netzwerkes (Verbindungen 2-er Knoten) und k die Anzahl Knoten, so müssen z linear unabhängige Gleichungen gefunden werden, um alle Grössen im Netzwerk zu berechnen.  \\Die Gleichungen können folgendermassen gefunden werden: \\
\formulaBegin
(k-1) Gleichungen können aus Knotengleichungen gefunden werden. \\
(z - (k-1)) übrige Gleichungen werden mithilfe von Maschengleichungen gefunden.
\formulaEnd
\iend


\vorgehen{Vorgehen}{Analyse von Umfangreichen Netzwerke}{}
\beginip
\begin{center}
	\begin{itemize}
		\item Entferne alle Widerstände und Quellen und zeichne das Schaltbild neu.
		\item Nummeriere alle Konten. Die Anzahl der Knoten wird mit k bezeichnet.
		\item Alle Verbindungslinien, welche zwei Knoten verbinden, werden als Zweige bezeichnet. \#Zweige = z
		\item Definiere Ströme für jeden Zweig und stelle (k-1) Knotengleichungen auf. Schreibe diese am besten bereits in Matrix schreibweise: \\
		      Bsp: $I_1 + I_2 = I_3, I_3 - I_4 = 2A $: \\
		      $
		      \left[ {\begin{array}{cccc}
		      		1 & 1 & -1 & 0 \\
		      		0 & 0 & 1 & -1 \\					\end{array} } \right] \left[ {\begin{array}{c} I_1 & I_2 & I_3 & I_4 \\ \end{array} } \right] =   \left[ {\begin{array}{c}  0 & 2A\\ \end{array} } \right] $ \\

		      			\item Finde z - (k-1) unabhängige Maschengleichungen. (Mittels vollständiger Baum / Auftrennen der Maschen)
		      			\item Ersetze die Spannungen der Maschengleichung mit Strom mal Widerstand und ergänze das Gleichungssystem.
		      			\\ Bsp:
		      			$U_1 + U_2 = 5V, U_2 - U_3 = U_4$ \\ $ \rightarrow R_1 \cdot I_1 + R_2 \cdot I_2 = 5V, R_2 \cdot I_2 - R_3 \cdot I_3 - R_4 \cdot I_4 = 0$ \\
		      			$
		      			\left[ {\begin{array}{cccc}
		      					1 & 1 & -1 & 0 \\
		      					0 & 0 & 1 & -1 \\
		      					\mathbf{R_1} & \mathbf{R_2} & 0 & 0 \\
		      					0 & \mathbf{R_2} & \mathbf{-R_3} & \mathbf{-R_4} \\
		      					\end{array} } \right] \left[ {\begin{array}{c} I_1 & I_2 & I_3 & I_4 \\ \end{array} } \right] =   \left[ {\begin{array}{c}  0 & 2A & \mathbf{5V} & \mathbf{0} \\ \end{array} } \right] $ \\

		      						\end{itemize}
		      						\end{center}
		      						\iend

	 \newpage
\section{Magnetostatik}
\subsection{Das Magnetische Feld und die Lorentzkraft}

Analog zum elektrischen Feld definieren wir ein magnetisches Feld, dessen Feldlinien Kräfte auf eine bewegte Ladung auswirken. \\
Auslöser für das magnetische Feld sind \textbf{bewegte Ladungen} welche gemäss der rechten Hand Regel ein Magnetfeld hervorrufen, das diese einschliesst. \\
\begin{center}

	\ibox{\includegraphics[scale=0.3]{img/hfeld.png}}

\end{center}
\definition{Magnetische Flussdichte \textbf{B}}
\beginip
Bewegte Ladungsträger welche sich in der Nähe anderer bewegter Ladungsträger befinden, verspüren eine Kraftwirkung welche senkrecht zur Bewegungsrichtung zeigt. \\
Analog zum elektrischen Feld definieren wir ein magnetisches Feld (= magnetische Flussdichte) welche diese Kraftwirkung beschreibt. \\
Die magnetische Flussdichte, weisst jedem Punkt im Raum einen Vektor zu. Die resultierende Kraft auf ein Ladungsträger, welcher sich mit Geschwindigkeit V bewegt,
berechnet sich aus dem Kreuzprodukt von Geschwindigkeit und Ladung ($\rightarrow$ \textbf{Lorentzkraft}). \\
Fliesst eine elektrische Flussdichte (B-Feld) \textbf{senkrecht} durch einen Materialübergang, so bleibt es konstant. \\
- Es scheint so, als wäre die Flussdichte unabhängig vom Material
\iend




\definition{Rechte Hand Regel}
\beginip
Die magnetische Flussdichte $B$ um einen stromdurchflossenen Leiter baut sich stets gegen den Uhrzeigersinn auf und ist \textbf{immer geschlossen}. \\
\begin{center}
	\ibox{\includegraphics[scale=0.2]{img/rechte-hand-regel.png}}
\end{center}
\iend

\newpage
\gl{Gleichung}{Lorentzkraft}
\begingl
Bewegte Ladungen in einem Magnetfeld verspüren eine Kraft, welche proportional zur Stärke des B-Feldes und der Geschwindigkeit ist. \\
Die Kraft steht senkrecht zu den Feld- und Geschwindigkeitsvektoren.


\begin{center}
	\ibox{\includegraphics[scale=0.3]{img/lorentzkraft.png}}
\end{center}

\formulaBegin
$\vec{F_m} = q \cdot \vec{v} \times \vec{B} = I(\vec{l} \times \vec{B})$
\formulaEnd

\textbf{Variabeln}: \\
$ F_m = $ Magnetische Kraft $ [F_M] = N$ \\
$ \vec{v} = $ Geschwindigkeit der Teilchen $ [v] = \frac{m}{s}$ \\
$ \vec{B} = $ Magnetische Flussdichte $ [B] = T (Tesla)$ \\
$ \vec{l} = $ Länge des Leiters im Magnetfeld $ [l] = m $\\

Die Summe mit der Coulomb Kraft bezeichnet die \textbf{Lorentzkraft} \\

\formulaBegin
$\vec{F_L} = \vec{F_m} + \vec{F_c} = q \cdot \vec{v} \times \vec{B} = I(\vec{l} \times \vec{B}) + q \cdot \vec{E} $
\formulaEnd

In älteren Physikbüchern wird häufig nur die magnetische Kraftwirkung als Lorentzkraft bezeichnet.

\iend

\newpage
\definition{H-Feld}
\beginip
Legen wir ein magnetisches Feld an eine Materie an, so richten sich die im Material vorhandenen magnetischen Dipole mit dem angelegten Feld aus und \texttt{"}stärken\texttt{"} dieses. \\
Das eigentliche Feld, welches wir anlegen, bezeichnen wir als Magnetische Feldstärke (= H-Feld). Das Feld, welches entsteht und sich aus H-Feld und Magnetisierung des Materiales (M-Feld) zusammensetzt, ist das B-Feld.

\formulaBegin
$\displaystyle \mu_0 ( \vec{H} + \vec{M})  =  \vec{B} $\\
$\rightarrow \vec{H} = \frac{\vec{B}}{\mu_0\cdot \mu_r}
$
\formulaEnd

\begin{center}
	\ibox{\includegraphics[scale=0.4]{img/h-feld.png}}
\end{center}
\iend

\gl{Gleichung}{Durchflutungssatz}
\begingl
Der Durchflutungssatz sagt etwas darüber aus, wie das magnetische Feld entsteht. \\
Es bringt die \textbf{Magnetische Feldstärke H} (Feld) und den \textbf{Strom I} (Quelle) in Verbindung.
Die Aussage des Durchflutungssatz ist, dass das Wegintegral des H-Feldes über eine beliebige \textbf{geschlossene} Kurve gerade dem Wert des durch diese Fläche fließenden Stromes entspricht. \\
Diesen Wert definieren wir als Durchflutung $\Theta$ \\
\formulaBegin
$\displaystyle \Theta = \oint_{s} \vec{H} \cdot d\vec{s} = \iint_{A} \vec{j} \cdot d\vec{A} = I_{eff}$
\formulaEnd
Dabei ist es wichtig, das nur der \textbf{effektiv durch eine Fläche hindurchfliessende Strom} die Magnetische Durchflutung auslöst. Fliesst in einer Fläche gleich viel Strom hinein wie hinaus, ist die Durchflutung gleich 0.
\iend

\gl{Gleichung}{Vereinfachter Durchflutungssatz}
\begingl
Für den Fall, dass der Strom N mal durch eine Fläche hindurchfliesst und das H-Feld auf dem gesamten Weg konstant und parallel zum Weg ist, gilt folgende Vereinfachung:
\formulaBegin
$\displaystyle \Theta = H \cdot l_s = N\cdot I \rightarrow H = \frac{N \cdot I}{l_s}$
\formulaEnd
\iend

\bsptask{Beispiel}{}
\beginbsp
Berechnen Sie die Durchflutung $\Theta$ und das H-Feld $\vec{H}$ um einen unendlich langen, mit Strom I durchflossenen Leiter. Der Radius des Leiters sei R. \\
\iend
\bsp{Lösung}{}
\beginbsp
Wir wählen als geschlossenen Weg einen Kreis mit Radius $\rho$ um unseren Leiter. \\
Da wir von einem perfekten Leiter ausgehen, treffen wir die Annahme, dass das Magnetfeld achsensymmetrisch und somit konstant entlang des Kreises ist. \\
\\
Wir erhalten: für  $ \rho > R$: \\
$\displaystyle  \doubleunderline{\Theta} =  \oint_{2\pi \rho} \vec{H} \cdot d\vec{S} = \doubleunderline{I} $ \\
$ \displaystyle \rightarrow |H| \cdot 2 \pi \rho = I \rightarrow \doubleunderline{\vec{H}(\rho) = \frac{I}{2 \pi \rho} \cdot \vec{e_{\varphi}}} $\\

Für  $ \rho < R $ : \\
$\displaystyle  \doubleunderline{\Theta(\rho)} =  \oint_{2\pi \rho} \vec{H} \cdot d\vec{S} =   \doubleunderline{\frac{\rho^2\cdot I}{R^2}} 		  $ \\
$ \displaystyle \rightarrow |H| \cdot 2 \pi \rho = \frac{I \cdot \rho^2}{R^2} \rightarrow \doubleunderline{\vec{H}(\rho) = \frac{I}{2\pi R^2} \cdot \rho \cdot \vec{e_{\varphi}}} $

\textbf{Skizze}
\begin{center}
	\ibox{\includegraphics[scale=0.7]{img/ex4-1.png}}
\end{center}
\iend

\bsptask{Beispiel}{Magnetfeld einer Spule}
\beginbsp
Gegeben sei eine Spule mit N Wicklungen und der länge l. Berechnen Sie die magnetische Feldstärke $\vec{H}$ in Abhängigkeit des Stromes I, Länge l und Wicklungen N.
\begin{center}

	\ibox{\includegraphics[scale=1.3]{img/ex-spule}}
\end{center}





\iend
\bsp{Lösung}{}
\beginbsp
Es gilt für die Durchflutung bezüglich des roten Weges:

\begin{center}
	\ibox{\includegraphics[scale=1.3]{img/spule-sol.png}}
\end{center}





\begin{center}

	$\displaystyle \Theta_{Spule} = \oint_{s} \vec{H} \cdot d\vec{s} = \iint_{A_s} \vec{j} \cdot d\vec{A} = N \cdot I \simeq \int_0^l \vec{H} \cdot d\vec{s}$
\end{center}

Da die Anordnung \textbf{symmetrisch zur X-Achse} ist, wird das H-Feld auf dem gesamten Weg konstant sein und in X-Richtung zeigen,
\begin{center}

	$ \displaystyle \Theta_{Spule} = N \cdot I = H \cdot l_{s}$

\end{center}
Und somit gilt für das H-Feld mit Richtungsvektor:
\begin{center}

	$\displaystyle H = \frac{N\cdot I}{l_{s}}  = \frac{N\cdot I}{l}$ \\ \fspace
	$\displaystyle
	\doubleunderline{   \vec{H}(\vec{r}) =
		\begin{cases}
			0                                              & $Aussherhalb der Spule $ \\
			\displaystyle \frac{N\cdot I}{l_{s}} \vec{e}_x & $Innerhalb der Spule $   \\
		\end{cases}}$

\end{center}
\iend


Aus der letzten Aufgabe folgt, dass das magnetische Feld einer Spule im Inneren Näherungsweise konstant ist. Ausserhalb der Windungen ist die Spule feldfrei. \\
Diese Eigenschaften entsprechen gerade denen eines \textbf{Stabmagneten}


\subsubsection{Verhalten von B und H Feld an Randflächen}
\fix \fix
\textbf{B-Feld} \\
Fliesst ein B-Feld durch einen Materialübergang, so verändert sich die \textbf{Tangentialkomponente}. Die \textbf{Normalkomponente} bleibt gleich.
\begin{center}
	\ibox{\includegraphics[scale=0.3]{img/brand}}
\end{center}

\formulaBegin
$ \displaystyle B_{n1} = B_{n2}$
\\ \fspace

$\displaystyle B_{t2} = B_{t1} \frac{\mu_2}{\mu_1}  = \frac{tan(\alpha_2)}{tan(\alpha_1)}$
\formulaEnd

\textbf{H-Feld} \\
Fliesst ein H-Feld durch einen Materialübergang, so verändert sich die \textbf{Normalkomponente}. Die \textbf{Tangentialkomponente} bleibt gleich.

\begin{center}
	\ibox{\includegraphics[scale=0.3]{img/hrand}}
\end{center}

\formulaBegin

$\displaystyle H_{n2} = H_{n1} \frac{\mu_1}{\mu_2} = \frac{tan(\alpha_1)}{tan(\alpha_2)} $ \\
\fspace
$ \displaystyle H_{t1} = H_{t2}$
\formulaEnd




\subsubsection{Das Reluktanzmodell}

\definition{Magnetische Spannung}
\beginip
Als magnetische Spannung bezeichnen wir das Wegintegral über die magnetische Feldstärke H.
\formulaBegin
$\displaystyle V_{M_{AB}} := \int_A^B \vec{H} \cdot d\vec{s}$
\formulaEnd

Die magnetische Spannung ist einzig eine Hilfsgrösse, um Werte zu berechnen. \\
Analog zum elektrischen Feld, lässt sich auch hier eine Maschengleichung definieren: \\
\begin{center}
	$\displaystyle \oint_{Masche} \vec{H} \cdot d\vec{s} = \Theta_{Masche}$
\end{center}
Falls wir nun die Durchflutung $\Theta$ als Quelle einfügen, gilt für ein magnetisches Netzwerk:
\begin{center}
	$\displaystyle \oint_{Masche} \vec{H} \cdot d\vec{s} + \Theta_{Quellen} = 0$
\end{center}
\iend

\definition{Magnetischer Fluss}
\beginip
Als magnetischen Fluss $\Phi$ bezeichnen wir die \texttt{"}Menge B-Feld\texttt{"}, welche durch eine gegebene Fläche fliesst. \\
\formulaBegin
$\displaystyle \Phi := \iint_A \vec{B}  \cdot d\vec{A} \simeq \pm B \cdot A, \ \ \ [\Phi] = T	\cdot m^2 $
\formulaEnd
\iend





\definition{Magnetischer Widerstand}
\beginip
Als magnetischer Widerstand $R_m$ bezeichnen wir das Verhältnis zwischen magnetischer Spannung und magnetischem Fluss \\
Er sagt etwas darüber aus, wie gross der magnetische Fluss bei einer gegebenen magnetischen Spannung ist.
\formulaBegin
$\displaystyle R_m = \frac{V_M}{\Phi} \underbrace{=}_{magn. Leiter} \frac{l}{\mu \cdot A}$
\formulaEnd
\iend

\textbf{Begründung} \\
Wir gehen davon aus, dass die magnetische Spannung über einem Leiter mit Länge l anliegt, dessen Querschnittfläche A ist. \\
\begin{center}
	$\displaystyle \frac{V_M}{\Phi} = \frac{\int_0^l \vec{H} \cdot d\vec{s}}{\iint_a \vec{B} d \vec{A}} = \frac{l \cdot H}{\mu \cdot A \cdot H} = \frac{l}{\mu \cdot A} $

\end{center}


\newpage

\textbf{magnetische Grössen im Vergleich zu eletrischen} \\

\def\arraystretch{2}%  1 is the default, change whatever you need
\begin{tabular}{c|c|c||c|c}
	                & Elektrisch                                                                                & Einheit                                               & Magnetisch                                                              & Einheit                             \\
	\hline
	\hline
	Leitfähigkeit  & $ \kappa $                                                                                & $\texttt{[}   \frac{1}{\Omega \cdot m}    \texttt{]}$ & $\mu (= \mu_0 \cdot \mu_r)$                                             & $\texttt{[}  \frac{H}{m}\texttt{]}$ \\
	Widerstand      & $ R = \frac{l}{\kappa A} $                                                                & $\texttt{[}   \Omega   \texttt{]}$                    & $R_m = \frac{l}{\mu A}$                                                 & $\texttt{[} \frac{1}{H}\texttt{]}$  \\
	Leitwert        & $ G = \frac{1}{R} $                                                                       & $\texttt{[}  S \texttt{]}$                            & $\Lambda_m = \frac{1}{R_m}$                                             & $\texttt{[}  H  \texttt{]}$         \\
	\hline


	Spannung        & $\displaystyle U_{AB} = \int_A^B \vec{E} \cdot d\vec{s}$                                  & $\texttt{[}V\texttt{]}$                               & $\displaystyle \Theta_{AB}= \int_A^B \vec{H} \cdot d\vec{s}$            & $\texttt{[}A\texttt{]}$             \\
	Strom / Fluss   & $\displaystyle I = \iint_A \vec{j}\cdot d\vec{A} = \kappa \iint_A \vec{E} \cdot d\vec{A}$ & $\texttt{[}A\texttt{]}$                               & $ \iint_A \vec{B} \cdot d \vec{A} = \mu \iint_A \vec{H} \cdot d\vec{A}$ & $\texttt{[}Wb\texttt{]}$            \\
	\hline
	Ohmsches Gesetz & $U = R \cdot I $                                                                          &                                                       & $\Theta = R_m \cdot \Phi $                                              &                                     \\
	Maschenregel    & $ U_0 = \sum_{Masche} U_m $                                                               &                                                       & $ \Theta(= NI) = \sum_{Masche} V_m $                                    &                                     \\
	Knotenregel     & $ \sum_{Knoten} I_k = 0 $                                                                 &                                                       & $ \sum_{Knoten} \Phi_k = 0 $                                            &                                     \\

\end{tabular}

\definition{Reluktanzmodell}
\beginip
Das Reluktanzmodel besagt, dass man eine magnetische Anordnung, als elektrisches Schaltbild beschreiben und berechnen kann. \\
\textbf{Vorgehen} \\
\begin{itemize}
	\item Spulen werden mit Spannungsquellen ersetzt $V_m = N\cdot I$
	\item Magnetkerne/Luftspälte etc. werden mithilfe der Länge und Querschnittsfläche als Widerstände modelliert. $R_m =  \frac{l}{\mu \cdot A}$
	\item Für magnetische Widerstände gelten die gleichen Regeln wie bei elektrischen (Seriellschaltung / Paralellschaltung).
\end{itemize}

\iend
\newpage
\bsptask{Beispiel}{}
\beginbsp
\textbf{Aufgabe Hubmagnet} \\
Der mittlere Schenkel (2) eines E-Kernes aus Dynamoblech trägt eine Wicklung mit N Windungen. Über
die drei Luftspalten mit gleicher Länge $\delta$ wird ein Anker aus Grauguss mit der Kraft FA angezogen
E-Kern und Anker besitzen die gleiche Dicke d. \\
\begin{center}
	\ibox{\includegraphics[scale=0.4]{img/ex5-1.png}}
\end{center}
Gegeben sind folgende Parameter: \\

\begin{center}
	\includegraphics[scale=0.6]{img/ex5-3.png}
\end{center}
Berechnen sie die magnetische Spannung auf dem Weg ACD.
\iend

\newpage
\important{Lösung}{}
\beginbsp
Zuerst zeichnen wir ein Reluktanzmodell des Magneten. \\
Wobei $R_L$ die Luftspälte, ${R_D}_i$ die Beine des Magneten und ${R_G}_i$ sowie $R_{BC}$ und $R_{CD}$ das Gusseinsenstück modellieren.


\begin{minipage}[t]{0.55\textwidth}
	\ibox{\includegraphics[width=0.9\columnwidth]{img/ex5-1-sol.png}}
\end{minipage}
\begin{minipage}[t]{0.05\textwidth}
	\begin{center}
		\vspace{-3cm}
		$\Rightarrow$ \\

		\fspace
		\fspace
	\end{center}
\end{minipage}
\begin{minipage}[t]{0.29\textwidth}
	\ibox{\includegraphics[width=1\columnwidth]{img/ex5-2.png}}
\end{minipage}



Für die Spannungsquelle erhalten wir:
\begin{center}

	$V_0 = N \cdot I_s = 1000 \cdot 211.3mA = 211.3A$ \\
\end{center}
Für die Widerstände:
\begin{center}
	$R_{D1} = R_{D3} = \frac{2b + 2a}{\mu_0 \mu_{rD} a d} = 79.6 \cdot 10^3H^{-1}$ \\
	$R_{D2} = \frac{b + \frac{a}{2} }{2 \mu_0 \mu_{rD} a d} = 17.9 \cdot 10^3H^{-1}$ \\

	$R_{L1} = R_{L3} = \frac{\delta}{\mu_0 a d} = 79.6 \cdot 10^3H^{-1}$ \\
	$R_{L2} =  \frac{\delta}{2\mu_0 a d} = 39.8 \cdot 10^3 H^{-1}$\\
	$R_{G1} = R_{G3} = \frac{\frac{a}{2}}{\mu_0\mu_{rG}ad} = 31.8 \cdot 10^3H^{-1}$ \\
	$R_{G2} =\frac{\frac{a}{2}}{2\mu_0\mu_{rG}ad} = 15.9 \cdot 10^3H^{-1}$ \\
	$R_{BC} = R_{CD} = \frac{b + \frac{3}{2}a}{2\mu_0\mu_{rG}ad} = 350.1\cdot 10^3H^{-1}$ \\
\end{center}
Weiter können wir die einzelnen Widerstände seriell zusammenfassen:
\begin{center}
	$R_1 = R_{D1} + R_{L1} + R_{G1} = 191 \cdot 10^3 H^{-1}$ \\
	$R_2 = R_{D2} + R_{L2} + R_{G2} = 73.7 \cdot 10^3 H^{-1}$ \\
	$R_3 = R_{D3} + R_{L3} + R_{G3} = 191 \cdot 10^3 H^{-1}$ \\
	$R_E = R_{BC} = R_ {CD} = 350.1 \cdot 10^3 H^{-1} $ \\
\end{center}

Die Spannung $U_{AC}$ lässt sich als Spannungsteiler berechnen:

\begin{center}
	$\displaystyle U_{AC} = U_0 \cdot \frac{((R_1 + R_E) || (R_3 + R_E)) } { ((R_1 + R_E) || (R_3 + R_E)) + R_2} = 166A$ \\
\end{center}
Und somit die Spannung $ U_{AD}$:
\begin{center}
	$\displaystyle \doubleunderline{U_{AD}} = 166A \cdot \frac{R_1}{R_1 + R_E} = \doubleunderline{58.6A}$
\end{center}
\iend

\newpage

\subsection{Spule und Induktivität}

\definition{Induktivität}
\beginip
Die Induktivität L beschreibt, wieviel magnetischer Fluss $\Phi$ sich bei einem Strom I im Inneren eines Bauteiles aufbaut.
\formulaBegin
$ \displaystyle L := \frac{N\cdot \Phi}{I} = \frac{N^2}{R_m} $
\formulaEnd
Der Faktor $N$ im Nenner kommt daher, dass der Fluss bei einer Spule durch $N$ Windungen hindurchfliesst. Somit ist die effektive Fläche des Flusses N mal grösser, wesshalb er N mal gezählt wird. \\

Die in einer Induktivität gespeicherte Energie berechnet sich zu
\formulaBegin
$W =\displaystyle \frac{1}{2}L \cdot I^2$
\formulaEnd
\iend



\bsptask{Beispiel}{Berechnen einer Induktivität}
\beginbsp
Auf einem Ringkern mit der Querschnittsfläche A und der Permeabilität $\mu_r \rightarrow \infty$ ist
eine Wicklungen mit $N_1$ Windungen angebracht. Der Ringkern
besitzt einen Luftspalt mit der sehr kleinen Breite $l_g$. Das magnetische Feld kann im
Luftspalt als homogen angenommen werden.

1) Berechnen sie die magnetische Flussdichte im Luftspalt. \\
2) Berechnen sie den magnetischen Fluss $\phi_A$ im Kern.\\
3) Berechnen sie die Induktivität L
\begin{center}

	\includegraphics[scale=0.25]{img/induktivitaet_bsp_1.PNG}

\end{center}

\iend
\newpage

\bsp{Lösung}{}

\beginbsp
\begin{itemize}
	\item 1) Es gilt für die Durchflutung bezüglich dem Kreisring:
	      \begin{center}
	      	$\Theta =  \oint_s \vec{H}\cdot d\vec{s} = N_1\cdot i_1(t)$
	      \end{center}
	      Da die Permeabilität des Magneten gegen unendlich strebt und das B-Feld bei senkrechten Materialübergängen konstant ist gilt: \\

	      \begin{center}
	      	$\displaystyle \oint_s \vec{H} d\vec{s} = \underbrace{\int_{M} \frac{\vec{B}}{\mu_0 \cdot \mu_r} d\vec{s}}_{=0} + \int_{L} \frac{\vec{B}}{\mu_0} d\vec{s} = \int_{L} \frac{\vec{B}}{\mu_0} d\vec{s} $
	      \end{center}
	      Da des B-feld parallel zum Weg ist und über dem ganzen Weg konstant ist gilt:
	      \begin{center}
	      	$\displaystyle N_1 \cdot i_1(t) = \displaystyle \oint_s \vec{H} d\vec{s} = \frac{B}{\mu_0} \cdot l_g$ \\
	      	$\displaystyle \rightarrow \doubleunderline{\vec{B} = \frac{N_1 \cdot i_1(t) \mu_0}{l_g} (-\vec{e}_x)}$
	      \end{center}

	\item 2) Für den magnetischen Fluss gilt:
	      \begin{center}
	      	$\displaystyle \phi_A = \iint_A \vec{B} \cdot d\vec{A} = B \cdot A =\doubleunderline{\frac{N_1 \cdot i_1(t) \mu_0}{l_g} \cdot A}$
	      \end{center}

	\item 3) Da die Spule $N_1$ Wicklungen aufweist, muss der Fluss $\phi_A$ genau $N_1$ mal gezählt werden. Es gilt für die Induktivität:
	      \begin{center}
	      	$ \displaystyle L:= {\phi}{i} = N_1 \cdot  \frac{N_1 \cdot i_1(t) \mu_0}{l_g} \cdot A \cdot \frac{1}{i_1(t)} =  \doubleunderline{\frac{N_1^2 \cdot \mu_0}{l_g} \cdot A}$
	      \end{center}
\end{itemize}

\iend




\definition{Serien und Parallelschaltung}
\beginip
Induktivitäten verhalten sich analog zu Widerständen: \\
\textbf{Serienschaltung}
\formulaBegin
$ L_{serie} = \sum_{i=0}^n L_i $
\formulaEnd

\textbf{Parallelschaltung}
\formulaBegin
$\displaystyle \frac{1}{L_{par}} = \sum_{i=0}^n \frac{1}{L_i} \Bigg\rvert L_{par} = (L_1 || L_2 )$
\formulaEnd
\iend
\newpage


\definition{Gegeninduktivität}
\beginip
Die Gegeninduktivität beschreibt, wie viel magnetischer Fluss durch eine \textbf{andere} Leiterschleife durchfliesst, abhängig
des Stromes in der ersten Schleife.
\formulaBegin
$\displaystyle L_{21} = N_2 \cdot \frac{\phi_{21}}{i_1}$
\formulaEnd
\begin{center}

	\includegraphics[scale=0.3]{img/gegenind}
\end{center}
\iend


\bsptask{Beispiel}{Berechnen einer Gegeninduktivität}
\beginbsp
Beim Ringkern aus dem vorherigen Beispiel wird nun eine 2. Spule mit $N_2$ Wicklungen hinzugefügt. Berechnen sie die Gegeninduktivität $L_{21}$ bezüglich der Spule mit den $N_2$ Windungen.
\begin{center}

	\includegraphics[scale=0.5]{img/induktivitaet_bsp_2.png}
\end{center}
\iend

\newpage


\bsp{Lösung}{}
\beginbsp

Die Gegeninduktivität beschreibt, wieviel magnetischer Fluss durch die 2. Spule fliesst abhängig des Stromes der ersten Spule:

Da der Fluss im Magneten selbst gerade $\Phi_{21}$ beträgt und dieser durch $N_2$ Leiterschleifen fliesst, gilt:
\begin{center}

	$\displaystyle  L_{21} := \frac{\Phi_{21}}{i_1} \cdot N_1 =  \doubleunderline{\frac{N_1\cdot N_2 \cdot \mu_0}{l_g} \cdot A}$
\end{center}
\iend






%	\textbf{Übersicht} \\
%	\\
%	\begin{tabular}{|c|c|c|c|c|}
%	\hline
%		\textbf{Energie} & 	\textbf{Strom und Spannung} & 	\textbf{ DC-Verhalten} & 	\textbf{ High-AC Verhalten*}& 	\textbf{ Admitanz*} \\
%		\hline 	\hline
%		 & & & & \\
%	    $ \displaystyle L =  \frac{N \Phi}{I} $ & $\displaystyle i_L(t) = \frac{1}{L} \int_0^t u_c(t) dt $ & Leerlauf & Kurzschluss & $ \displaystyle j\omega L$  \\
%		  $\displaystyle W =  \frac{1}{2} L I^2 $ & $\displaystyle u_L(t) = L \cdot \frac{d}{d t} (i_c) $ & \includegraphics[scale=0.5]{img/kurzschluss} &   \includegraphics[scale=0.4]{img/leerlauf}  &   \\
%			 & & & & \\
%			\hline
%	\end{tabular}

	 %----------------------------------------------------------------
%
%  File    :  thesis-style.tex
%
%  Author  :  Keith Andrews, IICM, TU Graz, Austria
%
%  Created :  27 May 93
%
%  Changed :  19 Feb 2004
%
% styling and technical implementation adopted 2011 by Karl Voit
%----------------------------------------------------------------
\newpage
\section{Zeitlich veränderliches Magnetfeld}
\subsection{Induktion}

%TODO
einführung

\gl{Gleichung}{Induktionsgesetz}
\begingl
\formulaBegin
  $\displaystyle \oint_s \vec{E}\cdot d\vec{s} = -\frac{d}{dt} \big ( \iint_{A_s} \vec{B} \cdot d\vec{A} \big )$
\formulaEnd
\textbf{Variablen} \\
$B = $ Magnetische Flussdichte $[B] = T$\\
$A_s = $ Vom Weg S aufgespannte Fläche $[A_s] = m^2$ \\

Falls B Feld Konstant auf Fläche und senkrecht:

\formulaBegin
  $\displaystyle  \oint_s \vec{E}\cdot d\vec{s} = -\frac{d}{dt} \big ( B_{eff} \cdot A_s \big ) $
\formulaEnd
\iend

\definition{Bewegungsinduktion}
\beginip
%TODO
\iend

\bsptask{Beispiel}{Bewegungsinduktion}
\beginbsp
%todo
TODO
\iend

\definition{Selbstinduktion}
\beginip
%TODO
todo
\iend


\definition{Gegentinduktion}
\beginip
%TODO
todo
\iend


\subsection{Charakteristische Gleichungen von Kapazität und Induktivität}
\gl{Gleichung}{Induktivität}
\begingl
%TODO
todo
\iend

\gl{Gleichung}{Kondensator}
\begingl
Der Zusammenhang zwischen Strom und Spannung am Kondensator ist wie folgt gegeben
\formulaBegin
$\displaystyle u_c(t) = \frac{1}{C} \int_0^t i_c(t) dt$ \\

$\displaystyle i_c(t) = c \cdot \frac{d}{d t} (u_c)$ \\
\formulaEnd
\iend


\textbf{Begründung} \\
Mit dem Wissen, dass Strom definiert ist als die Ladung pro Zeit $ \displaystyle \frac{dQ}{dt} = I$ folgt folgendes: \\
\begin{center}
	$\displaystyle \frac{\partial}{\partial t} (C \cdot U) = \frac{\partial}{\partial t} (Q) $ \\
	$ \displaystyle C  \cdot \frac{dU}{dt} = I \rightarrow U(t) = \frac{1}{C} \cdot \int_0^t I \cdot dt$
\end{center}

	 \newpage
	 \section{Übertrager}
\textit{Dieses Kapitel wurde aus dem PVK Skript von Gian Marti übernommen  }

\subsection{Gegeninduktion}
\begin{figure}[H]
	\center
	\vspace{-0.5cm}
	\includegraphics[width=0.65\textwidth]{img/Tra1}
	\vspace{-0.2cm}
\end{figure}
Zwei Leiterschleifen werden von zeitlich abhängigen Strömen $i_1(t), i_2(t)$ durchflossen (angeregt durch die Spannungen $u_1(t),u_2(t)$). Dabei erzeugen sie Flüsse $\Phi_{11}, \Phi_{22}$ in ihren jeweiligen Schleifen, aber auch Flüsse durch die jeweils gegenüberliegende Schleife, $\Phi_{12},\Phi_{21}$.
Nun kann man die Selbst- und Gegeninduktivitäten $L_{11},L_{22},L_{12},L_{21}$ defnieren als
$$L_{11} = \frac{\Phi_{11}}{i_1}, \qquad L_{22} = \frac{\Phi_{22}}{i_2}, \qquad L_{12} = \frac{\Phi_{12}}{i_2}, \qquad L_{21} = \frac{\Phi_{21}}{i_1}$$
dann gilt nach dem Induktionsgesetz (das Argument der Zeit, $t$, wird jeweils nicht ausgeschrieben):
\begin{equation*}
	\begin{alignedat}{2}
		u_1 &= R_1i_1 + \frac{d}{dt}\big(\Phi_{11}+\Phi_{12}\big) &&= R_1i_1 + L_{11}\frac{di_1}{dt}+L_{12}\frac{di_2}{dt} \\
		u_2 &= R_2i_2 + \frac{d}{dt}\big(\Phi_{22}+\Phi_{21}\big) &&= R_2i_2 + L_{22}\frac{di_2}{dt}+L_{21}\frac{di_1}{dt} \\
	\end{alignedat}
\end{equation*}
Die Zählrichtungen für die Flüsse $\Phi$ sind übrigens frei wählbar, die Vorzeichen der Induktivitäten $L$ folgen dann daraus. Die Gleichungen hier gelten für die Flüsse wie eingezeichnet. \newline

Es lässt sich zeigen, dass \textit{immer} $L_{ik} = L_{ki}$, also definieren wir $M=L_{12}=L_{21}$ sowie den Koppelfaktor
$k =  \pm\frac{M}{\sqrt{L_{11}L_{22}}}$ (das Vorzeichen hängt von von den gewählten Zählrichtungen ab) und schreiben
\begin{equation*}
	\begin{alignedat}{2}
		u_1 &= R_1i_1 + L_{11}\frac{di_1}{dt}+M\frac{di_2}{dt} &&= R_1i_1 + L_{11}\frac{di_1}{dt}+k\sqrt{L_{11}L_{22}}\frac{di_2}{dt} \\
		u_2 &= R_2i_2 + L_{22}\frac{di_2}{dt}+M\frac{di_1}{dt} &&= R_2i_2 + L_{22}\frac{di_2}{dt}+k\sqrt{L_{11}L_{22}}\frac{di_1}{dt}\\
	\end{alignedat}
\end{equation*}
\pagebreak

\subsection{Transformatoren}
\begin{figure}[H]
	\center
	\vspace{-0.5cm}
	\includegraphics[width=0.75\textwidth]{img/Tra2}
	\vspace{-0.2cm}
\end{figure}
Ein Transformator besteht aus (mindestens) zwei Wicklungen, die auf einem hochpermeablen Kern gewickelt und somit magnetisch eng gekoppelt sind (man kann idealerweise von einem streuungsfreien übertrager ausgehen). \newline

Wie bei der Gegeninduktion fomulieren wir die Induktivitäten aus:
$$ L_{11} = \frac{\color{red}\Phi_{11}}{i_1}, \qquad L_{22}= \frac{\color{green}\Phi_{22}}{i_2}, \qquad
M = \frac{\color{red}\Phi_{21}}{i_1}=\frac{\color{green}\Phi_{12}}{i_2}$$
(Die roten Grössen stehen für den von der linken Seite (Primärseite) erzeugten Fluss, die grünen für den von der rechten Seite (Sekundärseite) erzeugten.) Das Induktionsgesetz sagt:
\begin{equation*}
	\begin{alignedat}{2}
		u_0 &= R_1i_1 + u_1 &&= R_1i_1 \color{red} + L_{11}\frac{di_1}{dt} \color{green} - M\frac{di_2}{dt} \\
		u_3 &= R_2i_2 + u_2 &&= R_2i_2 \color{green} + L_{22}\frac{di_2}{dt} \color{red}-M\frac{di_1}{dt}\\
	\end{alignedat}
\end{equation*}
Die negativen Vorzeichen kommen daher, dass hier bei der gewählten Zählrichtung die Sekundärflüsse $\Phi_{12},\Phi_{21}$ den Primärflüssen $\Phi_{11},\Phi_{22}$ entgegengesetzt sind. \newline


Im Fall eines streuungsfreien übertragers (aller Fluss bleibt im Magnetkern) gilt für die Induktivitäten:
$$ L_{11}= N_1^2\frac{\mu A}{l}, \qquad L_{22}= N_2^2\frac{\mu A}{l}, \qquad M = N_1N_2 \frac{\mu A}{l}$$
Und die Verhältnisse der Spannungen direkt am Transformator sind
$$u_2 =-\frac{N_2}{N_1}u_1$$
Nun betrachten wir den Fall, wo $u_3=0$:
\begin{figure}[H]
	\center
	\includegraphics[width=0.75\textwidth]{img/Tra3}
	\vspace{-0.2cm}
\end{figure}
In dem Fall können wir unsere Spannungsgleichungen auch schreiben als
\begin{equation*}
	\begin{alignedat}{2}
		u_0 &= R_1i_1 + L_{11}\frac{di_1}{dt} - M\frac{di_2}{dt} &&= R_1i_1 + \big(L_{11}-M\big)\frac{di_1}{dt} - M \frac{d\big(i_2-i_1\big)}{dt}\\
		0 &= R_2i_2 + L_{22}\frac{di_2}{dt} - M\frac{di_1}{dt} && = R_2i_2 - M \frac{d\big(i_1-i_2\big)}{dt} + \big(L_{22}-M\big) \frac{di_2}{dt} \\
	\end{alignedat}
\end{equation*}
Aber das entspricht ja genau folgendem Ersatzschaltbild (das \textit{kein} physikalisch realisierbares Netzwerk darstellen muss, weil die eingezeichneten Induktivitäten ggf. negative Werte annehmen können):
\begin{figure}[H]
	\center
	\includegraphics[width=0.75\textwidth]{img/Tra4}
	\vspace{-0.2cm}
\end{figure}
Zur Punktkonvention: \newline
Die Punkte sollen den Wicklungssinn im ESB verdeutlichen, der sonst nicht ersichtlich wäre.
Auf der Primärseite kann der Punkt noch frei gewählt werden, auf der Sekundärseite muss er dann so sein, dass die Potentialdifferenz zwischen dem Wicklungsanschluss mit Punkt und dem ohne Punkt gleichzeitig positiv bzw. negativ ist wie auf der Primärseite. \newline
In der dreidimensionalen Anordnung überlegt man sich die Punkte mit der Lenz'schen Regel: $i_1$ verursacht den rechtswendig zugeordneten Fluss $\Phi_1$. Dann ist der induzierte Strom $i_2$ so gerichtet, dass sein rechtswendig zugeordneter Fluss $\Phi_2$ dem ersten Fluss $\Phi_1$ entgegenwirkt. Aus der Richtung von $i_2$ folgt über $u_2 = R_2 i_2$ die Richtung von $u_2$.\newline

\subsection{Der ideale Übertrager}
Wir haben oben schon gesehen, dass im streufreien Übertrager $$\frac{u_1}{u_2} = -\frac{N_1}{N_2} := -"u$$ gilt. Das Spannungsverhältnis von $u_1$ und $u_2$ entspricht also dem Übersetzungsverhältnis von $N_1$ und $N_2$ (Vorzeichen hängt von Zähl- und Wicklungsrichtungen ab). Wenn jetzt ausserdem noch die Permeabilität des Magneten gegen unendlich geht ($\mu_r \rightarrow \infty$), dann sprechen wir vom idealen Übertrager, und aus dem Durchflutungsgesetz folgt: $$\frac{i_1}{i_2}=\frac{N_2}{N_1} = \frac{1}{"u}$$
Das Stromverhältnis von $i_1$ und $i_2$ entspricht also gerade dem Kehrwert des Wicklungsverhältnisses (Vorzeichen wieder abhängig von Zähl- und Wicklungsrichtung). \newline

Daraus folgt auch, dass die links abgegebene Leistung $p_1$ gerade der rechtsseitig aufgenommenen Leistung $p_2$ entspricht, was intuitiv sofort Sinn macht:
$$p_1 = u_1i_1 = -"u u_2~\frac{i_2}{"u} = -u_2 i_2 = p_2$$
Ab jetzt schreiben wir für primärseitige Grössen den Index $p$ statt $1$ und für sekundärseitige Grössen $s$ statt $2$. Für den idealen Übertrager führen wir folgendes Schaltbild ein:
\begin{figure}[H]
	\center
	\includegraphics[width=0.75\textwidth]{img/Tra5}
	\vspace{-0.2cm}
\end{figure}
Wir haben nochmal zusammengefasst:
$$"u = \frac{N_p}{N_s},\qquad \frac{u_p}{u_s} = \frac{i_s}{i_p} = \pm"u, \qquad p_p=u_pi_p=u_si_s=p_s$$
Wir betrachten nochmal die frühere Anordnung, aber in der neuen Notation:
\begin{figure}[H]
	\center
	\includegraphics[width=0.65\textwidth]{img/Tra6}
	\vspace{-0.2cm}
\end{figure}
Um die galvanische Trennung zwischen Ein- und Ausgangsseite zu verdeutlichen, führen wir einfach einen idealen Übertrager mit Übersetzungsverhältnis $"u=1$ ein. Das hat auf das Verhalten des Netzwerks keinen Einfluss:
\begin{figure}[H]
	\center
	\includegraphics[width=0.65\textwidth]{img/Tra7}
	\vspace{-0.2cm}
\end{figure}
Wie müssen wir aber jetzt das ESB anpassen, wenn wir ein Übersetzungsverhältnis $"u \neq 1$ haben möchten? So:
\begin{figure}[H]
	\center
	\includegraphics[width=0.65\textwidth]{img/Tra8}
	\vspace{-0.2cm}
\end{figure}
Falls nun unser Transformator nicht ideal ist, können wir dem durch hinzufügen von parasitären Widerständen, welche die Verluste repräsentieren, Rechnung tragen:
\begin{figure}[H]
	\center
	\includegraphics[width=0.75\textwidth]{img/Tra9}
	\vspace{-0.2cm}
\end{figure}

\textbf{Widerstandstransformation}\newline
Wie sieht ein Widerstand auf der Sekundärseite eines idealen Übertragers von der Primärseite betrachtet aus?
\begin{figure}[H]
	\center
	\includegraphics[width=0.85\textwidth]{img/Tra10}
	\vspace{-0.2cm}
\end{figure}
Es ist
$$R_e = \frac{u_p}{i_p} = \frac{"u u_s}{\frac{i_s}{"u}}="u^2\frac{u_s}{i_s} = "u^2R_2$$
\pagebreak

	%\section{Anhang}
	%\subsection{Katalog S.9}



a) Um das Ersatzschaltbild zu berechnen, müssen 2 Grössen berechnet werden: \\
1) Innenwiderstand \\
2) Leerlaufspannung oder Kurzschlussstrom \\
\\
1) Für die Berechnung des Innenwiderstandes werden alle Quellen zu null Gesetzt. \\
D.h. Spannungsquellen $\rightarrow$ Kurzschluss, Stromquellen $\rightarrow$ Leerlauf \\
\textbf{Ersatzschaltbild}


\begin{center}
  \includegraphics[scale=1.5]{katalog/katalog-1/ir-1.png} \\
\end{center}
Da der 5R Widerstand kurzgeschlossen ist, wird niemals Strom durch ihn hindurchfliessen. Somit können wir ihn durch einen Leerlauf ersetzen. \\
\begin{center}
\includegraphics[scale=1.5]{katalog/katalog-1/ir-2.png} \\
\end{center}

2R und 3R liegen Seriell, somit können sie zu einem Widerstand der Grösse 5R zusammengefasst werden. \\
Dieser Widerstand ist wiederum parallel zu R, womit wir für den gesamten Widerstand und somit $R_E$ folgendes erhalten. \\
\begin{center}
  $R_E = (2R + 3R || R) = \frac{5R^2}{6R} = \frac{5}{6}R$
\end{center}

2) Nun müssen wir noch die Leerlaufspannung der Ersatzschaltung berechnen. Dazu wenden wir das Superpositionsprinzip an: \\
Zuerst berechnen wir die Spannung $U_{AB}$ zwischen den Klemmen A und B in Abhängigkeit der Spannungsquelle: \\


\begin{center}
\includegraphics[scale=1.5]{katalog/katalog-1/uu-1.png}

\end{center}
\newpage
Die Widerstände 2R und 3R sind seriell.
\begin{center}
    \includegraphics[scale=1.5]{katalog/katalog-1/uu-2.png}
\end{center}

Da die Widerstandände ($ 5R + R$ und $5R$) parallel sind, muss über beiden Ästen die Gleiche Spannung U abfallen. \\
Somit können wir die Spannungsteilerregel anwenden: \\
\begin{center}
    $U_{AB}^{(1)} = U \cdot \frac{R}{R + 5R} = U \cdot \frac{1}{6}$
\end{center}

Nun müssen wir noch die Spannung $U_{AB}^{(2)}$ in Abhängigkeit der Stromquelle berechnen: \\
Dazu setzen wir die Spannungsquelle zu 0:
\begin{center}
  \includegraphics[scale=1.5]{katalog/katalog-1/iu-1.png}
\end{center}
Der Widerstand $R_5$ wird wieder kurzgeschlossen.
\begin{center}
  \includegraphics[scale=1.5]{katalog/katalog-1/iu-2.png}
\end{center}
Die Widerstäde $2R$ und $R$ können Seriell zusammengefasst werden, wodurch jedoch die Klemmen verschwinden :
\begin{center}
  \includegraphics[scale=2.0]{katalog/katalog-1/iu-3.png} \\
\end{center}

Nun können wir mithilfe der Stromteilerregel den Strom durch den roten Widerstand berechnen:
\begin{center}
  $I_{Rot} = I \frac{3R}{3R + 3R} = \frac{I}{2}$
\end{center}

Dieser Strom fliesst durch die beiden Widerstände $R$ und $2R$ somit gilt für die Spannung über dem roten $R$ Widerstand und somit für die Spannung $U_{AB}^{(2)}$:
\begin{center}
  $U_{AB}^{(2)}  = U_R = I_{Rot}\cdot R = \frac{I\cdot R}{2}$
\end{center}

\newpage
Somit gilt für die Leerlaufspannung gemäss Superposition:
\begin{center}
  $U_{E} = U_{AB}^{(1)} + U_{AB}^{(2)} = \frac{U}{6} + \frac{I\cdot R}{2}$
\end{center}


b) Es gilt: $R_E = \frac{5}{6} \cdot 12 \Omega = 10\Omega$ und $U_E = 2V + 3A\cdot 6\Omega = 20V$ \\
Für $I_E$ gilt:
\begin{center}
  $I_E = \frac{U_E}{R_E} = \frac{20V}{10\Omega} = 2A$
\end{center}


c) Um die Leistung über dem Widerstand $R_2$ zu maximiere, schliessen wir zuerst das Lastnetzwerk an unsere Ersatzquelle an und ersetzen danach den Widerstand $R_2$ mit offenen Klemmen und Formen erneut das Netzwerk zu einer realen Quelle um. Aus der Vorlesung ist bekannt, dass die Leistung über $R_2$ genau dann maximal ist, wenn $R_2 = R_i$ gilt, wobei $R_i$ den Innenwiderstand gegenüber den Klemmen bezeichnet. \\
Die Aufgabe reduziert sich als darauf, den Innenwiderstand gegenüber den Klemmen zu berechnen.
\begin{center}
        \includegraphics[scale=2.0]{katalog/katalog-1/lr-2.png}
        \includegraphics[scale=2.0]{katalog/katalog-1/lr-1.png}
\end{center}

Um den Innenwiderstand zu berechnen setzen wir die Quellen zu 0 und formen das Netzwerk um, bis nur noch ein Widerstand vorhanden ist. \\
\begin{center}

      \includegraphics[scale=2.0]{katalog/katalog-1/lr-3.png}
\end{center}
Im ESB sind die Widerstände $R_E$ und $R_1$ parallel. Beide zusammen sind wiederum seriell zu $R_3$. Somit gilt für den Innenwiderstand: \\
$R_i = (R_E || R_1) + R_3$ \\
Um maximale Leistung an $R_2$ abzugeben, muss folgendes gelten:
\begin{center}
  $R_2 = R_i = (R_E || R_1) + R_3 \Rightarrow R_3 = R_2 - (R_E || R_1)$ \\
  $R_3 = 11.5 \Omega - (5\Omega || 20 \Omega) = 7.5 \Omega$
\end{center}

\newpage
d) Um den Spannungsabfall über $R_2$ zu berechnen, berechnen wir die Leerlaufspannung an den Klemmen:
\begin{center}
        \includegraphics[scale=2.0]{katalog/katalog-1/lr-4.png} \\
\end{center}

Da durch den Widerstand $R_3$ kein Strom fliesst, gilt für die Spannung $U$:
\begin{center}
  $ U = U_{R_1} - U_{R_3} = U_{R_1} - 0A \cdot R_3 = U_{R_1}$
\end{center}
Die Spannung über $R_1$ können wir mithilfe des Spannungsteilers berechnen: \\
\begin{center}
  $U_{R_1} = U_E \cdot \frac{R_1}{R_E + R_1} = 15V \cdot \frac{20\Omega}{25\Omega} = 12V$
\end{center}
Aus der Vorlesung ist bekannt, dass bei maximaler Leistungabgabe, die Spannung über dem Lastwiderstand gerade die hälfte der Leerlaufspannung beträgt. Somit gilt für die Spannung über $R_2$:
\begin{center}
  $U_2 = \frac{U}{2} = 6V $ \\
  $P_2 = \frac{U_2^2}{R_2} = \frac{36V^2}{11.5\Omega} = 3.13W$
\end{center}





\newpage



\subsection{Katalog S.15}


\begin{itemize}
  \item a) Da es sich beim Spannungsmessgerät um ein Messgerät mit unendlich hohem Widerstand handelt, dürfen wir davon ausgehen, dass zwischen den Klemmen A und B kein Strom fliessen kann. Somit können wir die Verbindung zwischen A und B als Leerlauf modelieren.
  \begin{center}
      \includegraphics[scale=2.5]{katalog/katalog-1/a2-1.png}
  \end{center}
  Da die beiden Widerstandsäste parallel gescchaltet sind, muss über beiden Ästen die gleiche Spannung abfallen:
  \begin{center}
    $U_e = U_{R_1} + U_{R_\vartheta} = U_{R_2} +U_{R_3}$
  \end{center}
  Somit können wir die Spannung über $R_\vartheta$ mithilfe des Spannungsteilers berechnen:
  \begin{center}
      $U_{R_\vartheta} = U_e \cdot \frac{R_\vartheta}{R_1 + R_\vartheta}$
  \end{center}
  Die Leistung über einem Widerstand ist definiert als:
  \begin{center}
    $P_R = U_R \cdot I_R = \frac{U_R^2}{R}$
  \end{center}
  Somit gilt für die Leistung über dem Widerstand $R_\vartheta$ ;
  \begin{center}
      $P_{R_\vartheta} = \frac{U_{R_\vartheta}^2}{R_\vartheta} = (U_e \cdot \frac{R_\vartheta}{R_1 + R_\vartheta})^2 \cdot \frac{1}{R_\vartheta} =  \frac{U_e^2 \cdot R_\vartheta}{(R_1 + R_\vartheta)^2} $
  \end{center}

  Um den Maximalwert dieser Leistung in Abhängigkeit des Widerstandes $R_\vartheta$ herauszufinden, leiten wir die Leistung nach $R_\vartheta$ ab und setzen sie zu 0:
  \begin{center}
    $\frac{d}{dR_\vartheta}(P_{R_\vartheta}) \stackrel{!}{=} 0 \rightarrow R_\vartheta = R_1 = 1k \Omega$
  \end{center}
  Die benötigte Temperatur berechnet sich zu:
  \begin{center}
    $R(\vartheta) = 1k\Omega (1 + \alpha (\vartheta - \vartheta_0)) \stackrel{!}{=} 1k \Omega$ \\
    $\Rightarrow \vartheta = \vartheta_0 = 20$\textdegree
  \end{center}
  Für die Spannung $U_e$ erhalten wir:
  \begin{center}
    $50mW \stackrel{!}{=} P_{R_\vartheta} = U_e^2 \cdot \frac{1k\Omega}{4k\Omega}$ \\
    $\rightarrow 200mW = U_e^2$ \\
    $ \rightarrow U_e = 14.14V$
  \end{center}
  \item b) Für die Spannung $U_m$ können wir folgende Masche aufstellen:
  \begin{center}
    $U_m = U_{R_\vartheta} - U_{R_3}$
  \end{center}
  Wobei wir $ U_{R_\vartheta}$ und $U_{R_3} $ mit dem Spannungsteiler berechnen können:
  \begin{center}
    $U_{R_\vartheta} = U_e \frac{R_\vartheta}{R_\vartheta + R_1} $ \\
      $U_{R_3} = U_e \frac{R_3}{R_2 + R_3} $
  \end{center}
  Somit gilt für $U_m$:
  \begin{center}
    $U_m = U_e \big(\frac{R_\vartheta}{R_\vartheta + R_1}  - \frac{R_3}{R_2 + R_3} \big)$
  \end{center}
  Mit der Bedingung, $U_m(\vartheta = \vartheta_0 = 0 $\textdegree$) = 0V$ erhalten wir:

  \begin{center}
      $0V= U_e \big(\frac{R(\vartheta_0 )}{R(\vartheta_0) + R_1}  - \frac{R_3}{R_2 + R_3} \big) = U_e \big(\frac{0.9 k\Omega}{1.9 k\Omega}  - \frac{R_3}{R_2 + R_3} \big)$ \\
      $\rightarrow \frac{R_3}{R_2 + R_3} = \frac{0.9 k\Omega}{1.9 k\Omega}  $ \\
      $\rightarrow R_3 = \frac{0.9}{1.9} \cdot (R_2 + R_3) $
  \end{center}

  Für die Leistung gilt:
  \begin{center}
    $P_{(R_2,R_3)} = \frac{U_e^2}{R_2 + R_3}  \stackrel{!}{=} 10mW$ \\
    $\rightarrow (R_2 + R_3) = \frac{U_e^2}{10mW} = 14400\Omega$
  \end{center}
  Somit gilt:
  \begin{center}
    $R_3 = \frac{0.9 k\Omega}{1.9 k\Omega} \cdot (14400\Omega) = 6821.05 \Omega $ \\
    $R_2 = 14400\Omega - R_3 = 7578.95 \Omega$
  \end{center}


  \item c) Wir bezeichnen mit $U_{ideal}$ die Spannung bei idealer Messung und mit $U_{err}$ die Spannung mit ungenauen Widerständen.
  \begin{center}
    $U_{ideal} = 12V \big( \frac{R_\vartheta}{R_\vartheta + R_1} - \frac{R_3}{R_3 + R_2}\big)$ \\
    $U_{err} = 12V \big( \frac{R_\vartheta}{R_\vartheta + R_1' } - \frac{R_3'}{R_3'+R_2'}\big)$
  \end{center}
  Die Differenz ist:
  \begin{center}
    $F(R_\vartheta)) = U_{ideal} - U_{err} = U_e \big (\frac{R_\vartheta}{R_\vartheta + R_1 } -\frac{R_\vartheta}{R_\vartheta + R_1'} - \frac{R_3}{R_3 + R_2} + \frac{R_3'}{R_3'+R_2'})$
  \end{center}
  Der Fehler wird maximal, wenn die Differenz stark negativ wird. Dies ist der Fall, falls $\frac{R_3'}{R_3'+R_2'} < \frac{R_3}{R_3+R_2}$ und $  \frac{R_\vartheta}{R_\vartheta + R_1'} > \frac{R_\vartheta}{R_\vartheta + R_1}$  gilt. Somt gilt für die Widerstände:

   \begin{center}
     $R_3' < R_3 \rightarrow R_3' = 0.9 \cdot R_3$ \\
      $R_2' > R_2 \rightarrow R_2' = 1.1 \cdot R_2$ \\
      $R_1' < R_1 \rightarrow R_1' = 0.9 \cdot R_1$
   \end{center}
  Um die Temperatur herauszufinden, leiten wir die Differenz nach $R_\vartheta$ ab:
  \begin{center}
    $\frac{d}{d R_\vartheta} (F(R_\vartheta)) \stackrel{TR}{=} U_e \cdot \big( \frac{R_1}{(R_1 + R_\vartheta)^2} - \frac{R_1'}{(R_1' + R_\vartheta)^2}\big)\stackrel{!}{=} 0$ \\
    $\stackrel{TR}{\rightarrow} R_\vartheta = \sqrt{R_1 \cdot R_1'} = 0.949 \cdot R_1 = 949 \Omega$ \\
    $\rightarrow (1 + \alpha (\vartheta - \vartheta_0)) = 0.949  $ \\
    $\rightarrow \vartheta = 9.8$\textdegree
  \end{center}
  Der Betrag des Fehler berechnet sich zu:
  \begin{center}
      $ U_{err} = 12V \big( \frac{R_\vartheta}{R_\vartheta + R_1'} - \frac{R_3'}{R_3' + R_2'} \big) = 1.07V$ \\
      $ U_{ideal} \stackrel{!}{=} 1.07V \rightarrow R_\vartheta = $
  \end{center}
\end{itemize}

\newpage

\subsection{Katalog S. ???}



a) Es gilt:
 \begin{center}
  $\iint _A \vec{J}(\vec{r}) \cdot d\vec{A} = I$.
\end{center}
Falls das J-Feld und die Fläche \textbf{senkrecht} sind und das J-Feld überall auf dieser Fläche  \textbf{gleich Gross} ist, vereinfacht sich dies zu:
\begin{center}
  $ A_{eff}(\vec{r}) \cdot J(\vec{r})  = I \Rightarrow J(\vec{r}) = \frac{I}{A_{eff}(\vec{r})}$ \\
\end{center}
Wobei $A_{eff}$ die effektiv vom Strom durchflossene Fläche bezeichnet.
\end{itemize}
Somit gilt für die Stromdichte im Messwiderstand:
\begin{center}
  $ J(r) = \frac{I}{d \cdot 2 \pi r}$ \\
\end{center}

und somit in Vektorform (Zylinderkoordinaten):
\begin{center}
  $ \vec{J}(r) = \frac{I}{d \cdot 2 \pi r} \cdot \vec{e}_{r}$ \\
\end{center}

\item b) Der Zusammenhang zwischen E-Feld und J-Feld ist gegeben als:
\begin{center}
  $\vec{E} = \frac{1}{\kappa} \cdot \vec{J}$
\end{center}
Somt gilt für das E-Feld:
\begin{center}
    $\vec{E}(r) =  \frac{I}{d \cdot 2 \pi r \cdot \kappa} \cdot \vec{e}_{r}$
\end{center}
Für die Spannung $U_{AB}$ gilt: \\
\begin{center}
  $U_{AB} = \int_A^B \vec{E}(r) \cdot d\vec{s} = \int_{\frac{D_{Innen}}{2}}^{\frac{D_{Aussen}}{2}} E(r) \cdot dr =  \int_{\frac{D_{Innen}}{2}}^{\frac{D_{Aussen}}{2}} \frac{I}{d \cdot 2 \pi r \cdot \kappa} \cdot dr = \underline{\underline{\frac{I}{\kappa \cdot d \cdot 2 \pi} \cdot ln(\frac{D_{aussen}}{D_{innen}})}}$
\end{center}
Für den Widerstand R gilt:
\begin{center}
$  R_{AB} := \frac{U_{AB}}{I} = \frac{1}{\kappa \cdot d \cdot 2 \pi} \cdot ln(\frac{D_{aussen}}{D_{innen}})$
\end{center}

Mit den Werten: $D_{aussen} = 2cm$, $D_{innen} = 5mm$ , $d = 3mm$ und $\kappa = 12 \cdot 10^3 \frac{S}{m}$ gilt:
\begin{center}
  $R_{AB} = \frac{1}{12 \cdot 10^3 \frac{S}{m} \cdot 3\cdot 10^{-3} m \cdot 2 \pi} \cdot ln(\frac{2 \cdot 10^{-2}}{3 \cdot 10^{-3}}) = 8.387m\Omega$
\end{center}
\item c) Wir betrachten beide Fälle (-3mm und +3mm):  \\
 +3mm: $\rightarrow R' = \frac{1}{12 \cdot 10^3 \frac{S}{m} \cdot 3\cdot 10^{-3} m \cdot 2 \pi} \cdot ln(\frac{2 \cdot 10^{-2} + 3 \cdot 10^{-3}}{3 \cdot 10^{-3}}) = 9.005m\Omega \rightarrow \Delta R =  |R-R'| = 0.618 m\Omega$\\
 -3mm: $\rightarrow R' = \frac{1}{12 \cdot 10^3 \frac{S}{m} \cdot 3\cdot 10^{-3} m \cdot 2 \pi} \cdot ln(\frac{2 \cdot 10^{-2} - 3 \cdot 10^{-3}}{3 \cdot 10^{-3}}) = 7.669m\Omega \rightarrow \Delta R =  |R-R'| = 1.336 m\Omega$\\
 Daraus Folgt: Maximaler Fehler bei $-3mm$. $R'$ ist dann $7.669m\Omega$ und für den Fehler gilt: $ \Delta R = 1.336 m\Omega$
\end{center}


\item d) Beim Messen gilt:
\begin{center}
 $\frac{U_{AB}}{R_{AB}} = I \rightarrow \Delta I =  |\frac{U_{AB}}{R_{AB}} - \frac{U_{AB}}{R'_{AB}} | $ \\
 $ F = \frac {\Delta I}{I} = \frac{\Delta I}{U_{AB}/R_{AB}} = |1 - \frac{R_{AB}}{R'_{AB}}| = |1 - \frac{8.387m\Omega}{ 7.669}| = 9.28\% $
\end{center}


\item e) es gilt: $ U_{AC} = \int_{r_a}^{r_c} E \cdot dr$ und  $ U_{CB} = \int_{r_c}^{r_b} E \cdot dr$.  \\
Das Integral $\int_{r_a}^{r_b} E \cdot dr$ haben wir bereits ausgerechnet. Es ergibt:
$ \frac{I}{\kappa \cdot d \cdot 2 \pi} \cdot ln(\frac{r_b}{r_a}) $ \\
Somit lautet die Gleichung:
\begin{center}
  $ \frac{I}{\kappa \cdot d \cdot 2 \pi} \cdot ln(\frac{r_c}{\frac{D_{innen}}{2} }) = \frac{I}{\kappa \cdot d \cdot 2 \pi} \cdot ln(\frac{\frac{D_{aussen}}{2}}{r_c })$ \\
  $ \Rightarrow \frac{2 r_c}{D_{innen}} = \frac{D_{aussen}}{2 r_c} \rightarrow r_c = \frac{\sqrt{D_{aussen} \cdot D_{innen}}}{2} = 5mm$
\end{center}




												%%%% end{document}
\end{document}
%% vim:foldmethod=expr
%% vim:fde=getline(v\:lnum)=~'^%%%%\ .\\+'?'>1'\:'='
%%% Local Variables:
%%% mode: latex
%%% mode: auto-fill
%%% mode: flyspell
%%% eval: (ispell-change-dictionary "en_US")
%%% TeX-master: "main"
%%% End:
7
